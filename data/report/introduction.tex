\chapter{Introduction}\label{chapter:introduction}


\section{Background}

\subsection{History of aviation}
On December 17th, 1903, the Wright brothers made the first sustained and
controlled flight with a powered, heavier-than-air, aircraft, and in the years
after developed the first practical fixed-wing aircraft, the Flyer III. Their
invention of aircraft controls was a fundamental breakthrough and marked the
beginning of the Pioneer Era of aviation.

The Pioneer Era, lasting until the First World War in August 1914, saw flight
becoming an established technology. Aircraft exhibitions were held,
demonstrations given, and prizes with the intention of encouraging aviation were
offered. Plenty of developments took place in construction, configuration,
controls, propellers and engines. Centres were established for aeronautical
research and flying schools were opened. In 1911 the first aircraft was used for
military purposes by Italy in the Italian-Turkish war and soon after they were
also deployed during the First World War.

In the period between the World Wars airplanes evolved from low-powered biplanes
made from wood and fabric to high-powered monoplanes made of aluminum. Many
aviation firsts occured during this period, like the first transatlantic flight
in 1919, and commercial airlines followed soon on routes like these.

Development and production continued at an ever higher pace during the
the Second World War and saw the development and deployment of jet aircraft
as well as the first turboprop engine that went into mass-production.

After the Second World War commercial aviation grew rapidly, with the first
purpose-built commercial jet airliner scheduled into service in 1952, and the
first sustained and regular service airline operating only 4 years after.
In 1947 a rocket-powered aircraft went through the sound-barrier and
this quickly led to the development of supersonic interceptor aircraft as a
countermeasure against long-range bombers. The development of intercontinental
ballistic missiles and the succesful launch of the Sputnik 1 began the Space
Race increasing again the pace of aeronautical developments.

The year 1969 saw the use of military Vertical/Short Takeoff and Landing
(V/STOL) aircraft, the first humans on the moon, the unveiling of the Boeing
747, and the maiden flight of the Concorde supersonic passenger airliner.
Commercial airliners started using higher bypass-ratios, resulting in better fuel
economy and less noise. \todo{rewrite part}

In the last quarter of the \nth{20} century, the Digital Age, emphasis changed.
Digital computers were used for design and modelling, and digital systems also
started to appear inside the aircraft. Digital fly-by-wire systems improved
manoeuvrability, stability and drag. In-flight management of the flight plan was
being handled by the newly introduced flight management system, reducing the
workload of the crew.

The beginning of the \nth{21} century saw the application of autonomous unmanned
aerial vehicles (UAVs). The first entirely autonomous flight across the Atlantic
became reality and the solar-powered plane Solar Impulse 2 completed a
circumnavigation of the Earth.

\subsection{Urbanisation, transportation and the impact of aviation}
In 2014 54\% of the world population was living in urban areas. % and this proportion is expected to increase to 66\% by 2050 \cite{UnitedNations2014}.
The world population is growing rapidly and it is expected that the given
percentage increases to 66\% by 2050 \cite{UnitedNations2014}, resulting
in higher urban densities.

People not only live closer to eachother than ever before, but also transport
more than before. Transport allows trade and greater spread of people and is
therefore an important aspect for economic growth. People typically commute to
where they work or study, and transport themselves as well for leisure.
Passenger transport is the essence of tourism, and commerce requires the
transport of people to conduct business. Production and consumption of goods and
products can occur at different locations, thus requiring transport.

% https://en.wikipedia.org/wiki/Environmental_impact_of_aviation
While there is a demand for transport, transport also has a negative
impact on our environment.
Aviation contributes to climate change and air and noise pollution, and these
environmental issues have the potential to limit the operation and growth of
airports. Indeed, aircraft noise is already a limiting factor for the capacity of
regional and international airports throughout the world \cite{Zaporozhets2011}.

Being a major user of energy, transport burns most of
the world's petroleum creating air pollution and contributes to global warming and thereby climate change.
Despite more fuel-efficient and less polluting turbofan and turboprop engines,
the rapid growth of air travel in recent years contributes to an increase in
total pollution attributable to aviation. In the European Union, greenhouse gas
emissions from aviation increased by 87\% between 1990 and 2006. By 2020,
aviation emissions are likely to more than double from present levels
\cite{European2006}.
Because of increasing densification the amount of people living nearby airports
increases as well, and these people are affected by air and noise pollution.


% Transportation has negative impacts on the environment.

% At the beginning of the 20th century the first practical aircraft were developed
% and since then developments in aviation went fast.


\subsection{Aircraft noise and human response}
Aircraft noise is noise associated with the operation of airports and in
particular the nature and extent of noise exposure rising from aircraft
operations. It is the single most significant contemporary environmental
constraint for aviation, and is likely to become more severe in the future
\cite{Zaporozhets2011}.

Aircraft noise produced during operations in the vicinity of airports represents
a serious social, ecological, and economic problem. The noise has a negative
impact on people's health, lowers their quality of life, and lessens their
productivity at work \cite{Zaporozhets2011}.



%
%
% Aircraft noise can cause annoyance and sleep disturbance. Currently, annoyance
% and sleep disturbance are predicted using indicators based on time-averaged sound pressure
% levels. To obtain a more complete representation of annoyance one should
% predict the audible aircraft sound and determine the impact of the sound on people.
%
% \subsection{Human response}
%
% Annoyance and sleep disturbance
%
% \subsection{Noise policy}
%
% World Health Organisation
%
%
% Noise control / mitigation
%
% Environmental Noise Policy (END)
%
% Zurich Aircraft Noise Index (ZFI) \cite{Schaffer2012}
%
%
%
% \subsection{Noise prediction}
%
%
%
% \subsubsection{Prediction models}
%
%
% Noise-Power-Distance-tables, or NPD-tables, are the backbone of a wide range of aircraft noise models in use, like the Integrated Noise Model (INM) \cite{Arntzen2011}.
%
% At Empa the noise prediction model FLULA is used, and a new model called sonAIR is in development \cite{sonAIR}.
%
%
%
%
%
% \subsection{The urban environment soundscape}

% \begin{quotation}
% In 450BC Confucius is reputed to have said: "Tell Me and I Will Forget; Show Me and I May Remember; Involve Me and I Will Understand."
% \end{quotation}



% The urban environment contains many different kinds of sounds, coming from sources ranging from cars to cats, ... to the music someone is listening to in his or her living room whilst a window is open.

% Soundscape


\subsection{Auralisation}% of soundscapes}
% \begin{quote}
% Auralisation is the technique of creating audible sound files from numerical data
% \end{quote} \cite{Vorlander2008}.

% TODO: taken from overview paper
% Auralisation is a technique to simulate the audible sound of an object or environment.
Auralisation is a method to render audible virtual sound fields \cite{Kleiner1993}.
The method has been used to simulate the audible sound inside
rooms, but also for the sound of
cars \cite{Forssen2009,Maillard2012,Pieren2015,Hoffmann2016,Hoffmann2016a},
trains \cite{Pieren2016},
windturbines \cite{Pieren2014,Heutschi2014},
fans \cite{Merino2016} and
aircraft \cite{Arntzen2014a, Rizzi2016a, Rizzi2016}. Auralisation of outdoor sources
and environments is a key topic of the VASTCON Technical Working Group \cite{Vastcon}.

% Common reasons for generating auralisations is to determine how new and old spaces sound,

Different auralisation methods exist. Often emission synthesis and sound
propagation are separated. This is not always the case because its not always
possible, like e.g. when using a wave-based method
\cite{Hornikx2016,Georgiou2016,Georgiou2016a}.

Common methods for the emission synthesis are spectral modelling synthesis (SMS)
and granular synthesis. Granular synthesis typically uses grains based on
measurements and is a computationally fast method for synthesis. Spectral
modelling synthesis generates a signal through a superposition of two types of
signal components: tones and noise. Whereas with the granular synthesis method
each grain typically corresponds directly to specific conditions, e.g. the speed
of and distance to the source, with spectral modelling synthesis the synthesis
strategy can be considered separate from the underlying model, and therefore a
emission synthesis model can be established that relates tonal and spectral
components to the operational state of the source.



\subsection{Aircraft noise auralisation}

% TODO: taken from overview paper
Aircraft auralisation has been used to study future aircraft types
\cite{Rizzi2013,Rizzi2016,Rizzi2016a} and flight procedures \cite{Sahai2016} but
also to investigate the perceived unpleasantness of aircraft flyover noise as
function of certain temporal parameters \cite{Pate2017}.

Aspects to consider when simulating the sound of aircraft are the different
noise sources on the aircraft, the state of the aircraft and thereby the state
of these sources, as well as the condition of the environment.
% \subsection{Aircraft emission}
The main noise sources of a turbojet aircraft are jet noise, fan and turbine
noise, combustion noise and airframe noise. In the case of turboprop aircraft
the main source is the propellor\cite{Zaporozhets2011}. Fan or propellor noise
is mostly tonal whereas the other sources are broadband noise.
Which exact sources are most relevant depends on the aircraft type and flight
procedure as well as the position of the source with respect to the receiver due
to directivity of the sources \cite{Bertsch2015}.

The aircraft emission prediction tools found in the ANOPP-Source Functional
Module of ANOPP2 \cite{Lopes2016, Tuttle2017} and INSTANT\cite{Sahai2016b}, which is based
on ANOPP, use established models for the noise prediction of the individual
noise sources. The Heidmann model is for example used for fan noise and the
Stone model for jet noise. The Heidmann model in ANOPP models
five sources explicitly, of which three correspond to emission of tones and two
to emission of broadband noise \cite{Arntzen2014a}. The model outputs for each
of these five sources a spectrum in fractional-octaves.
For broadband synthesis in the NAF\cite{Aumann2015}, power of the tonal
components in each band is divided by the amount of tones in that band.
Nowadays the Heidmann model in ANOPP can output the frequencies and amplitudes
of forward and aft radiated fan tones. Only Buzz-Saw noise is still output in
\nicefrac{1}{3}-octaves.


%
% % TODO: clean up the following
% \todo{cleanup the following}
% To enhance future aircraft operations, the Dutch National Aerospace Laboratory (NLR) wants to have a more comprehensive picture on aircraft noise annoyance.
% A cooperation of the NRL with NASA focuses on expanding the annoyance research capabilities \cite{Arntzen2011}. The NLR and NASA use a common setup, the Virtual Community Noise Simulator (VCNS).
% A helmet with visor is used to present a visual simulation and headphones are used to present binaural flyover noise.
%
% At NASA, Rizzi et al. auralised the flyover noise of a hybrid wing body (HWB) aircraft as well as a reference aircraft, similar to a Boeing 777-200ER.
%
% % To predict an aircraft trajectory the flight mechanics are modelled, instead
%
% At RWTH Aachen University a virtual reality room was developed with... \cite{Schroder2010}
%
%
% Figure \ref{fig:background_aircraft_noise_spectrogram_original} shows a spectrogram of a recorded aircraft passage.
%
% \begin{figure}[H]
%         \centering
%         \includegraphics[width=0.9\textwidth]{../figures/spectrogram_of_measurement}
%         \caption{Spectrogram of a measured aircraft passage. Doppler shifted tones can clearly be seen.}
%         \label{fig:background_aircraft_noise_spectrogram_original}
% \end{figure}
%


\newpage
\section{Aim}

% \subsection{Aim}

The aim of the thesis is to develop a tool to simulate the audible sound of
aircraft in an urban environment, so that new methodologies can be developed to
assess annoyance and sleep disturbance due to aircraft noise.

There's a large variation of different aircraft in use nowadays, that could each
be perceived differently. An additional constraint is therefore that the tool
should be able to simulate the audible sound of the current fleet of aircraft.


\todo{Check!}

The aim of my research is to develop a tool for the auralisation of aircraft noise in 
an urban environment. This tool can then be used to determine better annoyance 
and sleep disturbance indicators and can then be used in conjunction with a noise prediction 
model like sonAIR to give better a better estimate on how many people are affected by aircraft noise and how severe.

The auralisation tool has to support a typical urban situation where shielding 
and reflections may play an important role and should include correct over-head 
directional information. The work is therefore divided in three parts:

\begin{itemize}
 \item Development of an aircraft source synthesiser
 \item Development of propagation model
 \item Installation of reproduction system for auralisation.
 \end{itemize}

% \section{Thesis structure}
\section{Outline}
The thesis is structured as follows.
\newline
\newline
Chapter \ref{chapter:theory} provides an overview of theory required to understand the concepts that are used throughout the thesis.
Discussed the basics of sound, signal processing, auralisation, previous work on aircraft auralisation and emission synthesis, and finally atmospheric turbulence.
\newline
\newline
Chapter \ref{chapter:tool} describes the auralisation tool that was developed. The propagation model is explained as well as the algorithm that is used to extract features from recordings and synthesise aircraft sounds based on that.
\newline
\newline
Chapter \ref{chapter:turbulence} gives an extensive overview on a novel algorithm for simulating fluctuations due to atmospheric turbulence and how to apply this algorithm in auralisations.
\newline
\newline
Chapter \ref{chapter:test} aims to answer the question whether the auralisation tool provides sufficiently plausible auralisations
\newline
\newline
Finally, the work is summarised in Chapter \ref{chapter:conclusions} and future work is suggested.


\section{Limitations}



