Aircraft noise is a major issue in urban areas and is one of the research topics
within the FP7 SONORUS project. Current methods for determining the impact of
aircraft noise on annoyance and sleep disturbance are based on energetic
quantities disregarding the dynamic character of the sound. To obtain a more
complete representation of annoyance, listening tests with audible aircraft
sound promise to determine the impact of the aircraft sound on people more accurately.

A tool was developed to auralise aircraft noise. The propagation model includes,
aside from the typically considered propagation effects like Doppler shift, a
computationally efficient model to generate sequences of amplitude and phase
fluctuations due to atmospheric turbulence. The method, already used in the
field of wireless communication to predict the performance of wireless
communication links, could be used in the field of acoustics to create more
perceptually valid auralizations. Depending on parameters that are used to
describe the atmospheric turbulence, the amplitude modulations result in clearly
audible modulations and peaks in a spectrogram. Furthermore, the phase
modulations cause additional decorrelation.

An inverse propagation model was developed to compute back from a receiver to
source in time-domain. An automated procedure was developed to extract features
from the resulting signal. These features were then used to directly synthesise
the emission as function of time, and this signal was propagated to the original
receiver resulting in an auralisation that should reproduce the recording
it is based on.

To determine whether the auralisations indeed resembled the recordings, and thus
to validate the auralisation tool, a listening test was conducted. Participants
were asked to rate the similarity of two sounds in a paired comparisons. Two
aircraft types were considered, and the participants were clearly able to
discriminate between them, in case of both the recordings and the auralisations.
However, according to the participants the auralisations did not yet fully match
the recordings.
\todo{rewrite}




% OLD

% An increasing amount of people live in urban environments. The amounts of
% traffic increase as well, a While aircraft have become more quiet over the
% years, there is also more people than ever flying, and thus the amount of people
% exposed to aircraft noise increases.
%
% Noise has a negative impact on humans, it can cause annoyance, distraction,
% sleep disturbance, stress. The impact of sound on humans depends on multiple
% aspects, like the character of the sound as well as expections from those
% exposed to it. Current aircraft noise assessment methods are based on metrics that consider
% only the energy of the sound and not the character. To better predict the impact
% of sound on humans, the audible sound should be considered instead of just a level.
%
% Recordings can be used to investigate the human response to aircraft sound,
% however, with recordings you're limited in the parameters you can very, and thus
% in how you can control a human response experiment. Auralization is a process to
% render a soundfield audible. While originally used to predict the sound inside a
% space, it is now also used to simulate the soundscape of outside spaces.
% Depending on the auralization method one can adjust parameters affecting
% sound generation and propagation. \todo{controlled experiment}
%
% Auralizations have been made of cars, windturbines and aircraft. Typically the
% source synthesis is done using Spectral Synthesis Modelling and the sound
% propagation is based on geometrical acoustics. Spectral Synthesis Modelling
% allows for a detailed sound synthesis, with the limiting factor typically being
% the lack of sufficiently detailed input.
%
% \todo{emission synthesis}
%
% Propagation models typically assume a steady atmosphere as this significantly
% simplifies the modelling as the resulting system is entirely deterministic. In
% practice, atmospheric turbulence, which is of a stochastic nature, affects the
% refractive index of the medium and thus wave propagation.
% % In practice, spatial and temporal fluctuations of windspeed and temperature, that
% % are of a stochastic nature, affect the refractive index of the medium and thus
% % wave propagation. \todo{maybe drop refractive index}
% The effects of atmospheric turbulence on wave propagation are clearly audible.
% It is therefore expected that atmospheric turbulence affects the perceived
% realism of aircraft auralizations and is therefore required to study the impact
% of aircraft sound on humans.
%
% Modelling sound propagation through a turbulent atmosphere is typically
% computationally expensive. Therefore, a simple and fast model was developed
% to account for atmospheric turbulence in auralizations. \todo{something about model}

% \todo{listening tests}
%
% \todo{conclusions}

\vspace{0.1cm}

\textbf{Keywords}: Aircraft noise, Auralization, Outdoor sound propagation, Atmospheric turbulence
