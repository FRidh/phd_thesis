\section*{Introduction}
Aircraft noise can cause annoyance and sleep disturbance. Currently, annoyance 
and sleep disturbance are predicted using indicators based on time-averaged sound pressure 
levels. To obtain a more complete representation of annoyance one should 
predict the audible aircraft sound and determine the impact of the sound on people.

A new noise prediction model is being developed at Empa called SonAIR. 
My task at Empa is to develop a tool for the auralisation of aircraft noise in 
an urban environment. This tool can then be used to determine better annoyance 
and sleep disturbance indicators and can then be used in conjunction with 
SonAIR to, for example, give better estimates of how many people are affected by 
aircraft noise.

The auralisation tool has to support a typical urban situation where shielding 
and reflections may play an important role and should include correct over-head 
directional information. The work is therefore divided in three parts:

\begin{itemize}
 \item Development of an aircraft source synthesiser
 \item Development of propagation model
 \item Installation of reproduction system for auralisation.
 \end{itemize}


\section*{Development of an aircraft noise source synthesiser}
The aircraft noise source synthesiser synthesises the sound close to the source.
In earlier work \cite{Arntzen2011} an aircraft trajectory was modelled using the 
flight mechanics instead of a radar track.
Based on the flight mechanics and several models (Stone, Heidmann, Finkel) to 
predict spectral components a source signal was created.
One challenging type of noise to include is Buzz-Saw noise. This type of noise 
is generated by shocks on a fan when the tip is moving supersonic. 
While work is done at better predicting this type of noise for future turbofans 
\cite{McAlpine2007} there does not yet exist a model that relates 
turbofan/aircraft types with conditions to emitted Buzz-Saw noise.

These models are made for use in noise predictions models, and it may not be so 
straightforward to use these models for auralisation.
Therefore \textbf{an overview will be made on how to synthesise an aircraft noise 
signal}.

In September a large measurement campaign was conducted at our division 
collecting sound recordings of landings as well as take-offs at several 
locations. Along with detailed information regarding flight conditions this opens the 
possibility to \textbf{develop an empirical model describing Buzz-Saw noise as 
function of aircraft and condition} as well as other types of noise.
Challenge will be to discriminate between the types of noise as well as 
atmospheric influences.



\section*{Development of propagation model}
First of all, it will be assumed that the source and propagation can be 
described independently of each other. In an urban environment reflections and 
shielding may play an important role and therefore these effects have to be 
included. Other effects that might be relevant are metrological effects.

The propagation model that is being developed supports reflections by using 
mirror sources that are found using the Image Source Method. Wind and temperature 
gradients cannot be accounted for but it has been shown that these effects are 
generally small and can be neglected \cite{Arntzen2012}. An effect that cannot 
be neglected is amplitude modulation and decoherence due to atmospheric turbulence.

The idea is to \textbf{develop an empirical model for a time-varying digital filter describing 
the modulation and decoherence based on atmosphere/turbulence parameters}.
Numerical modelling will be done of the turbulent field and sound field in time-domain.


\section*{Installation of reproduction system for auralisation}
For the reproduction of the auralisation a system has to be set up. One issue to 
cope with is correct over-head information. There exists a localisation error 
when using an Ambisonics system instead of real sources. 

In an earlier study it was found that the error in elevation angle is 20 to 30 degrees for first and 
second order Ambisonics systems and still 10 to 20 degrees for third order 
systems \cite{Power2013}. This is an issue that is likely relevant to look at 
and it might even be necessary to consider other reproduction methods like e.g. Wave Field Synthesis if it is 
not possible to reproduce elevated sources correctly using Ambisonics.


% \section*{Other ideas...}
% \begin{itemize}
%  \item Auralise indoor sound by including transfer path between facade and inside and room impulse response.
%  \item Use more detailed time-domain method in the receiver area and compare.
% \end{itemize}




