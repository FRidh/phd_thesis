Cities are expanding and more people than ever live close to an airport. An
airport nearby can be convenient, but there is also a strong disadvantage:
noise. Aircraft are loud, and while modern aircraft are quieter, the amount of
people exposed to aircraft noise has gone up significantly.

Aircraft noise is known to cause stress, annoyance, and sleep disturbance. In
order to determine how many people are affected by aircraft noise, we first need
to investigate when they are affected. Let us consider for example annoyance.
When is the aircraft noise causing annoyance and what is making it so annoying?
We need to find out what factors play a role, and then try to quantify their
contribution, because that gives the possibility to develop aircraft that sound
less annoying, or could cause less sleep disturance.

In order to investigate these aspects, experiments need to be conducted. This
thesis describes a tool that simulates the audible sound of aircraft in order to
investigate aspects like annoyance. A listening test was done in order to
evaluate the tool. The outcome was that the auralisations did not sound
sufficiently similar to the recordings yet, and improvements are therefore
needed before the tool could be used for investigating aircraft noise impact.

How realistic auralisations sound, can depend on many factors, and one factor that is
thought to be important is the impact of atmospheric turbulence on sound
propagation. When you listen to distant aircraft you can hear strong
fluctuations in the loudness. Earlier aircraft sound simulators did not include
this effect. An algorithm that would account for these fluctuations was
developed and implemented in the simulation tool.
