Aircraft noise is a major issue in urban areas. Due to a rising level of
urbanisation and the continuing growth of air traffic more people are exposed to
aircraft noise than ever. Methods currently used for assessing the
impact of aircraft noise on humans consider mostly energetic quantities, and not the
dynamic character of the sound. Therefore, in order to obtain a more accurate
picture of the impact of aircraft sound it may be helpful to assess how the audible
sound is perceived.

Auralisation is a method for rendering audible sound fields and may be used to
create audible aircraft sound. A tool was developed to auralise the sound of jet
airplanes and consists of an outdoor sound propagation model and an emission synthesiser.
The emission synthesiser computes an emission signal consisting of tonal
components and broadband noise. The spectral components vary over time and take
into account directivity.

% TODO rewrite
An inverse propagation model was developed to compute back from a receiver to
source in time-domain. An automated procedure was developed to extract features
from the resulting signal. These features were then used to directly synthesise
the emission as function of time, and this signal was propagated to the original
receiver resulting in an auralisation that should reproduce the recording
it is based on.

% TODO rewrite
To validate the auralisation tool, a listening test was conducted where participants
were presented with recordings and auralisations and had to rate their similarity.
% Two aircraft types were considered, and two events per type.
Results indicate that differences exist between the auralisations and recordings.
Improving the synthesis of the blade passing frequency is expected to improve the similarity
between auralisations and recordings.

Finally, fluctuations can typically be noticed when listening to sound from a
distant aircraft, and one cause of these fluctuations is atmospheric turbulence.
A computationally fast algorithm was developed to take into account the
amplitude and phase modulations that arise as the sound propagates through the
turbulent atmosphere. According to the author the method results in improved
plausibility of the auralisations.



\vspace{0.1cm}

\textbf{Keywords}: Aircraft noise, Auralisation, Outdoor sound propagation, Atmospheric turbulence
