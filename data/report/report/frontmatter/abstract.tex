Aircraft noise is a major issue in urban areas. Due to a rising level of
urbanisation and the continuing growth of air traffic more people are exposed to
aircraft noise than ever. Methods that are currently used for determining the
impact of aircraft noise humans consider energetic quantities and not the
dynamic character of sound. Listening tests with audible aircraft sound promise
to determine the impact of aircraft sound on people more accurately.

Auralisation is a method for rendering audible sound fields and may be used to
create audible aircraft sound. A tool was developed to auralise the sound of jet
airplanes and consists of a propagation model and an emission synthesiser based
on spectral modelling synthesis. Geometrical acoustics is considered and the
propagation model accounts for the typical propagation effects like spreading,
Doppler shift and atmospheric attenuation. Reflections are taken into account
using the Image Source Method.

The emission synthesiser computes an emission signal consisting of tonal
components and broadband noise. The spectral components vary over time and take
into account directivity. As input to the centerfrequencies, bandwidths, and
levels are needed.

% TODO rewrite
An inverse propagation model was developed to compute back from a receiver to
source in time-domain. An automated procedure was developed to extract features
from the resulting signal. These features were then used to directly synthesise
the emission as function of time, and this signal was propagated to the original
receiver resulting in an auralisation that should reproduce the recording
it is based on.

% TODO rewrite
To determine whether the auralisations indeed resembled the recordings, and thus
to validate the auralisation tool, a listening test was conducted. Participants
were asked to rate the similarity of two sounds in a paired comparisons. Two
aircraft types were considered, and the participants were clearly able to
discriminate between them, in case of both the recordings and the auralisations.
However, according to the participants the auralisations did not yet fully match
the recordings.


Finally, fluctuations can typically be noticed when listening to sound from a
distant aircraft. One cause of these fluctuations is atmospheric turbulence. In
order to improve the plausibility of the airplane auralisations a
computationally fast algorithm was developed to take into account the amplitude
and phase modulations that arise as the sound propagates through the turbulent
atmosphere. Sequences of modulations are computed and applied to each
propagation path. The modulations are a function of source-receiver distance,
transverse speed of the source, as well as parameters used to describe turbulent
field. Demonstrated are the effects of the method and these parameters on a
pure-tone signal as well as aircraft sound.



\vspace{0.1cm}

\textbf{Keywords}: Aircraft noise, Auralization, Outdoor sound propagation, Atmospheric turbulence
