\section{Atmospheric turbulence}

The theory of turbulence is a statistical theory.
The wind velocity components and temperature in the turbulent atmosphere are fluctuating functions of both position and time.

Due to these inhomogeneities scattering and decorrelation occur.



\missingfigure{Refractive-index field}

\missingfigure{Time signal showing fluctuations}

\subsection{Correlation function and refractive-index field}


% When listening to aircraft noise, sound level fluctuations caused by atmospheric turbulence are clearly
% audible. Therefore, to create a realistic auralisation
% of aircraft noise, atmospheric turbulence needs to be
% included.
% 
% Due to spatial inhomogeneities of the wind veloc-
% ity and temperature in the atmosphere, acoustic scat-
% tering occurs, affecting the transfer function between
% source and receiver. Both these inhomogeneities and
% the aircraft position are time-dependent, and there-
% fore the transfer function varies with time resulting
% in the audible fluctuations.
% The theory of turbulence is a statistical theory. A
% statistical theory fits well with the physics of turbu-
% lence since turbulence is a consequence of instability
% of fluid flow in relation to very small fluctuations in
% the fluid [5]. For an auralisation instantaneous values
% of the sound pressure at the receiver are required.
% Artnzen[1] included a phase fluctuations filter in
% their aircraft noise simulator to make the ground ef-
% fect less pronounced. The filter was based on a Gaus-
% sian spectrum of turbulence.


% \subsection{Correlation function}
The fields of the wind velocity components and the temperature are random fields.
A characteristic of a random function or field is it's correlation function. \cite{Tatarskii1971}
The correlation function of such a random field $f(\mathbf{r})$, as function of distance $\mathbf{r}$ between observation points only, is defined as
\begin{equation}
 B(\mathbf{r}_1, \mathbf{r}_2) = \langle f(\mathbf{r}_1  f(\mathbf{r}_2) \rangle
\end{equation}
Note that we've assumed a 'frozen turbulence' here. 

In a homogeneous and isotropic random field the correlation function $B(\mathbf{r})$ 
depends only on $|r|=\mathbf{r}$, i.e., only the distance between the observation points.

% \begin{figure}[H]
%         \centering
%         \includegraphics[]{../figures/ipynb/theory_atmospheric_turbulence_gaussian_covariance/figure1}
%         \caption{Correlation for a spherical wave as function of time lag.}
%         \label{fig:theory_atmospheric_turbulence_correlation}
% \end{figure}


\subsection{Gaussian spectrum}
In the region of large eddy sizes the turbulence spectrum can be approximated by a Gaussian distribution.
The spatial correlation function for the fluctuating refractive-index $\mu$ therefore also has a Gaussian form
\begin{equation}
 \langle \mu_1 \mu_2 \rangle = \langle \mu^2 \rangle \exp{\left( -x^2 / L^2 \right)}
\end{equation}
When the outer length scale of the turbulence $L$ is much smaller than the Fresnel zone, 
the mean square log-amplitude and phase fluctuations for spherical waves are given by
\begin{equation}\label{eq:model_daigle}
 \langle \chi^2 \rangle = \langle S^2 \rangle = \frac{\sqrt{\pi}}{2} \langle \mu^2 \rangle k^2 r L 
\end{equation}
where $k$ is the wavenumber and $r$ is the distance of propagation.
The mean square log-amplitude and phase fluctuations are defined respectively as $\langle \chi^2 \rangle = \langle \left( \ln{\frac{A_{n}}{A_{m}}} \right) ^2 \rangle $
and $ \langle S^2 \rangle = \langle \left( \phi_n - \phi_m \right)^2  \rangle$. The index $n$ indicates the sample $n$ at time $t$ and index $m$ the average fluctuation.
For spherical waves the covariances of the fluctuations, $B_{\chi}(\rho)$ and $B_{S}(\rho)$, normalized to their variances, are given by
\begin{equation}
 \frac{B_{\chi} (\rho)}{\langle \chi^2 \rangle} = \frac{B_{S} (\rho)}{\langle S^2 \rangle} = \frac{\Phi\left(\rho/L\right)}{\rho/L}
\end{equation}
where 
\begin{align}
 \phi \left(\rho/L \right) &= \int_0^{\rho/L} \exp{\left(-u^2\right)} \mathrm{d} u 
 &= \frac{1}{2} \sqrt{\pi} \mathrm{erf}\left( \rho/L \right)
\end{align}
with $\mathrm{erf}$ the error function.
The covariance of the fluctuations $B_{\chi}(\rho)$ and $B_{S}(\rho)$ are thus given by
\begin{equation}
 B_{\chi} (\rho) = B_{S}(\rho) = \frac{\sqrt{\pi}}{2}  \langle \mu^2  \rangle k^2 r L 
\frac{\Phi(\rho/L) }{\rho / L}
\end{equation}
according to Daigle's model.

\subsection{Time series of fluctuations}
% The fluctuations of the random fields cause amplitude and phase fluctuations or modulations.
For the auralisation, time series of instantaneous fluctuations of log-amplitude and phase are required, $\chi(t)$ and $S(t)$.
Because the modulations are frequency-dependent, we would like to obtain realisations of $\chi(t, f)$ and $S(t, f)$ which describe the fluctuation of the log-amplitude $\chi$ and phase $S$ at a time $t$ for frequency $f$.
The frequency $f$ is the frequency of the signal component that should be modulated.

By assuming that $\chi(t,f)$ and $S(t,f)$ are Gaussian random variables, we can produce realisations of $\chi(t,f)$ and $S(t,f)$ by filtering white noise with a chosen filter function.
The desired filter response is the impulse response $h(\rho)$ of the covariance $B(\rho)$, which is obtained by taking the inverse Fourier transform of the square root of the autospectrum of $B(\rho)$.
% \begin{equation}
%  h(\rho) = \sqrt{S_{\rho}}
% \end{equation}

By taking these steps, two series of fluctuations are obtained, $\chi(t, f)$ and $S(t, f)$. 
The two time series can be merged into a single complex modulation signal
\begin{equation}
 m(t,f) = \exp{\left( \chi\left(t,f\right) + j S\left(t, f\right) \right)}
\end{equation}
that can then be applied to the input signal.

\subsection{Saturation of the log-amplitude}
For longer path lengths and stronger turbulence, the amplitude fluctuations gradually level off.
Saturation of the amplitude fluctuations can be observed when measuring aircraft noise at distances of over a few kilometers.
The standard deviation of the fluctuating sound pressure levels is in such cases limited to no approximately 6 dB \cite{Daigle1983,Piercy1974}.

Saturation of the log-amplitude can be included based on an analysis by Wenzel \cite{Wenzel1975}.
In Wenzel's approach the soundwave is split up in a coherent and incoherent wave. The amplitude of the coherent wave decreases over distance while the incoherent wave obtains the energy from the coherent wave.
The coherent wave $p$ is written as
\begin{equation}
 \langle p \  p^* \rangle = \left( A_m^2 / 4 \pi r^2 \right) \exp{\left( -2 \langle \mu^2 \rangle k^2 r L \right)}
\end{equation}
Wenzel defines the distance to saturation $r_s$ as the distance at which the coherent wave is reduced to $e^{-1}$ of its original strength
\begin{equation}\label{eq:saturation_distance}
 r_s = \frac{1}{2 \langle \mu^2 \rangle k^2 L}
\end{equation}
Saturation of the log-amplitude fluctuations can now be included by multiplying $\chi(t,f)$ with
\begin{equation}
 \sqrt{ \frac{ 1}{1 + r/r_s}}
\end{equation}


% \subsection{Applying the modulations to the original signal}
% In case the input signal consists of just one frequency $f$ and the modulation
% signal $m$ is calculated for that specific frequency, then the modulation can simply be
% applied by multiplying the two in the time-domain.
% However, for the auralisation we have broadband sounds with more than one frequency line to consider.
% As the fluctuations are frequency-dependent, the modulation
% signal would have to be calculated for every frequency line.
%
% One of the design criteria of the auralisation tool is to produce at or near
% real-time auralisations. Since computing the modulation signal for every
% frequency line is computationally relatively heavy, another solution is sought.
% The variances of the log-amplitude and phase fluctuations increase according to
% this simple theory linearly with frequency. Therefore, another option would be to scale both
% $\chi(t,f)$ and $S(t,f)$ accordingly. Care should be taken considering the saturation distance is frequency-dependent.
