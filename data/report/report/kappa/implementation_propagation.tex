\section{Propagation model}

\subsection{Introduction}
The propagation model covers sound propagation in the far field through a
homogeneous and isotropic atmosphere. Sound is modelled as rays. The propagation
model takes into account geometrical spreading, Doppler shift, atmospheric
attenuation and the ground reflection. These propagation effects are taken into
account as signal processing operations.
A model was developed to include scintillations, that is, fluctuations in the
sound due to atmospheric turbulence \cite{Rietdijk2017}. That part is discussed
in detail in chapter \ref{chapter:turbulence}.

Figure \ref{fig:propagation_block_diagram} shows a block diagram of the steps
that are taken. Because of source motion all propagation effects are
time-variant. The order of the operation matters although some could be
exchanged. The Doppler-shift occurs between the source and the medium and
therefore divergence is computed first. The shifted frequencies are then
attenuated in the atmosphere. A reflection is modeled as a parallel path with a
filter representing the reflection coefficient. In case there is no movement
this filter may be considered part of the mirror source, however, as movement is
considered it needs to be placed after the Doppler-shift has been applied. The
following sections will describe each of these steps in detail.

% TODO remove turbulence from block diagram?
\begin{figure}[H]
  \centering
\begin{tikzpicture}[auto, node distance=1cm,>=latex']
\tikzset{
block/.style    = {draw, shape=rectangle, fill=white, minimum height=4em, minimum width=5em, text width=5em, align=center},
}
    % Main nodes
    \node [block]                       (emission)      {Emission\\Generator};
    \node [block, right=of emission]    (spreading)     {Spreading\\Gain};
    \node [block, right=of spreading]   (delay)         {Spreading\\Delay};
    \node [block, right=of delay]       (attenuation)   {Attenuation\\Filter};
    \node [block, right=of attenuation] (turbulence)    {Turbulence\\Filter};
    \node [block, right=of turbulence]  (reflection)    {Reflection\\Filter};

    % Main edges
    \draw [->]  (emission)      --  (spreading);
    \draw [->]  (spreading)     --  (delay);
    \draw [->]  (delay)         --  (attenuation);
    \draw [->]  (attenuation)   --  (turbulence);
    \draw [->]  (turbulence)    --  (reflection);

\end{tikzpicture}
  \caption{Block diagram of the propagation model. These steps are performed for each propagation path and the resulting signals are summed.}
  \label{fig:propagation_block_diagram}
\end{figure}

% TODO convective amplification is a source motion effect, not propagation.
% Discuss in other section?
% Convective amplification is supported in the
% propagation model but was ignored here.

% TODO explain situation - check noise signal!!
To demonstrate the different propagation effects a fixed situation is considered
and propagation effects are added sequentially. A simple
emission signal is considered consisting of brown noise and a tone with a
frequency of \SI{1000}{\hertz}, both modelled as a monopole point source. The
\say{aircraft} moves at a height of \SI{200}{\meter} and with a speed of
\SI{100}{\meter\per\second} straight over the receiver.
Figure \ref{fig:implementation:propagation:emission} shows a spectrogram of the
emission signal.

\begin{figure}[H]
  \centering
  \includegraphics[]{../figures/generated/propagation/emission}
  \caption{Spectrogram of the immission. Because in this case we do not consider sound propagation this in facts corresponds to the emission of the aircraft as function of the orientation. An omni-directional source is considered and therefore the emission is stationary.}
  \label{fig:implementation:propagation:emission}
\end{figure}


\newpage
\subsection{Geometrical spreading}
Geometrical spreading causes a decrease in amplitude with distance due to
divergence and increase in propagation delay with distance due to the finite
speed of sound. Far field is assumed in which case the sound pressure at the
receiver $p_{\mathrm{rcv}}$ is given by the Green's function given in equation
\eqref{eq:theory:sound:green-free-field}. The emission signal $p_{\mathrm{src}}$
is computed at a fixed distance from the source $r_{\mathrm{src}}$. The sound
pressure at a receiver at distance $r_{\mathrm{rcv}}$ can be obtained by
rescaling the magnitude of the sound pressure of the emission signal with the
ratio of the distances
\begin{equation}
 p_{\mathrm{rcv}} = p_{\mathrm{src}} \frac{r_{\mathrm{src}}}{r_{\mathrm{rcv}}}
\end{equation}
This operation is implemented as a simple gain. Figure
\ref{fig:implementation:propagation:spreading} shows a spectrogram of the
immission when geometrical spreading is taken into account.

\begin{figure}[H]
  \centering
  \includegraphics[]{../figures/generated/propagation/spreading}
  \caption{Spectrogram of the immission when the amplitude decrease due to geometrical spreading is included.}
  \label{fig:implementation:propagation:spreading}
\end{figure}

% \subsubsection{Geometrical spreading - phase}
The time-dependent propagation delay, which is relevant for the Doppler shift,
is taken into account by resampling the discretized sound pressure signal with a
Variable Delay Line. Since the signal is discrete and the delay is generally not
an integer multiple of the sample time, an interpolation scheme is required.

% TODO some of it needs to go to theory
As explained in section \ref{sec:theory:signal:resampling}, multiple
interpolation schemes exist, including Lanczos interpolation that was used for
auralisation as well \cite{Rietdijk2015, Pieren2015}. In this case a linear
interpolation scheme was chosen due to its simplicity and computational
performance \cite{Heutschi2014}. While linear interpolation can cause audible
artifacts, these are in the case of a broadband signal less pronounced than if
solely a tonal component is considered as was demonstrated in Figure
\ref{fig:theory:signal-processing:resampling}.

For a given sound travel time $\Delta t(t)$ from source to receiver, the non-integer sample index
$n_{e}$ of the signal $y[n_e]$ at the source time-axis is given as
\begin{equation}
 n_{e} = n_r - \Delta t (t) \cdot f_s
\end{equation}
where $f_s$ is the fixed sampling frequency and $n_r$ the integer sample index
of the signal at the receiver time-axis. Equation
\eqref{eq:theory:signal:resampling:interpolation-linear} can then be used for
applying the Doppler shift when using $n=\floor{n_e}$ and $\eta = n_e - \floor{n_e}$
where $\floor{n_e}$ corresponds to the floor function of $n_e$.
Furthermore, an emission time should exist and therefore there is the
requirement $\floor{n_e} >= 0$.

% The received signal value $y$ at index $n_e$ is then determined by
% \begin{equation}
%  y[n_e] = \left( 1 - n_e +  \floor{n_e} \right) \cdot y \left[ \floor{n_e} \right] + \left( n_e - \floor{n_e} \right) * y \left[ \floor{n_e} + 1 \right]
% \end{equation}
% where $\floor{n_e}$ corresponds to the floor function of $n_e$.
The sound travel time was computed with the following expression for the speed of sound $c =
343.2 \sqrt{ \frac{T}{T_0} }$ where $T$ is the temperature during the event and
$T_0 = 293.15$ K the reference temperature.

Figure \ref{fig:implementation:propagation:doppler} shows a spectrogram of the
immission when the propagation delay is also considered.

\begin{figure}[H]
  \centering
  \includegraphics[]{../figures/generated/propagation/doppler}
  \caption{Spectrogram of the immission when the propagation delay, and thus the Doppler shift, is included as well. The black part in the first couple of seconds is due to the initial propagation delay.}
  \label{fig:implementation:propagation:doppler}
\end{figure}

\newpage
\subsection{Atmospheric attenuation} % TODO refer to theory
Atmospheric attenuation is accounted for by creating a time-variant filter of length $N_{aa}$.
A single-sided magnitude spectrum is calculated as
\begin{equation}
 \left| H_{aa} \right| = 10^{- d \alpha(f) / 20}
\end{equation}
where $d$ is the source-receiver distance in \SI{}{meter} and $\alpha(f)$ the
frequency-dependent attenuation coefficient in \SI{}{\decibel\per\meter} computed using
equation \ref{eq:theory:sound:atmospheric-attenuation}.

% The air pressure, relative humidity, and temperature were recorded during the event.
The impulse response of a magnitude-only or zero-phase filter is non-causal and therefore in
order to create a causal filter a linear-phase filter corresponding to 90
degrees is added. The spectrum is real and even, and therefore the impulse
response is real and even as well. After convolving the signal with the designed
filter the first $N_{aa}/2$ samples were removed to account for the delay caused
by the linear-phase factor.

Convolution was performed using an overlap-save algorithm. The filter length was
4096 samples and the hop size 256 samples. Transitioning to the next impulse
response was done without smoothing because the impulse responses are relatively small.

Figure \ref{fig:implementation:propagation:attenuation} shows a spectrogram of
the immission when atmospheric attenuation is included. A reference atmosphere
was considered.

\begin{figure}[H]
  \centering
  \includegraphics[]{../figures/generated/propagation/attenuation}
  \caption{Spectrogram of the immission when atmospheric attenuation is included as well.}
  \label{fig:implementation:propagation:attenuation}
\end{figure}

% \subsection{Atmospheric turbulence} % TODO this overlaps with next chapter. Remove. Small parts can be reused.
% Fluctuations in the atmosphere of wind velocity and temperature affect sound
% propagation resulting in fluctuations of both the amplitude and the phase.
% Depending on the situation these fluctuations can be clearly audible and
% therefore have to be included in order to produce realistic sounding
% auralisations.
%
% A method for including phase fluctuations utilizing the mutual
% coherence function was presented in \cite{Arntzen2014a, Arntzen2014b, Shin2006}.
% The phase fluctuations cause decorrelation between the direct and
% ground-reflected contributions resulting in less-pronounced interference
% and generally more plausible auralisations.
%
% The authors of this paper developed a model that includes both log-amplitude and phase fluctuations due to
% atmospheric turbulence in auralisations \cite{Rietdijk2014, Rietdijk2014a,
% Rietdijk2017}. As the aircraft moves through the turbulent atmosphere, the
% refractive-index fluctuations are sampled by the waves propagating from source
% to receiver. The model considers a line-of-sight situation and assumes a frozen
% atmosphere. Other assumptions were a Gaussian applied turbulence spectrum and
% spherical wavefront \cite{Wilson1999,Daigle1983}. The model takes into account
% saturation of the log-amplitude fluctuations. Details about the model and how it
% can be used in auralisations can be found in \cite{Rietdijk2017}.
%
% The autocorrelation function of the Gaussian spectrum is given by \cite{Daigle1983}
% \begin{equation}
%  C = \frac{\Phi\left(\rho/L\right)}{\rho/L}
% \end{equation}
% and determines the shape of the log-amplitude $\chi$ and phase fluctuations $S$
% spectra. Because of the Wiener-Khinchin theorem we can take the type-1 Discrete
% Cosine Transform (DCT-1) to obtain the shape of the autospectra.
%
% The autocorrelation is a function of $\rho$, the spatial separation between two receivers,
% perpendicular to the wave propagation direction, and $L$, the correlation
% length. Instead of two non-moving receivers we consider a moving source
% % that samples the refractive-index fluctuations,
% and therefore perform a space-to-time conversion to obtain $C(\tau)$ and $\rho=v_{\bot}\tau$ where
% $v_{\bot}$ is the speed of the aircraft perpendicular to the wave propagation
% direction.
%
% Gaussian white noise is shaped with this spectrum through a convolution to
% obtain log-amplitude and phase fluctuations. Time-variance of the speed is taken
% into account not by updating the filter but instead by resampling the sequence of
% fluctuations using a Variable Delay Line with a delay based on the time-varying transverse speed.
%
% At this point the fluctuations have a variance of 1. The fluctuations are scaled
% by the square root of the desired variances \cite{Daigle1983}
% \begin{equation}\label{eq:variances}
%  \sigma_{\chi}^2 = \sigma_{S}^2 = \frac{\sqrt{\pi}}{2} \sigma_{\mu}^2 k^2 d L
% \end{equation}
% which are a function of the variance of the refractive-index fluctuations
% $\sigma_{\mu}^2$, wavenumber $k$, source-receiver distance $d$ and correlation
% length $L$. This scaling can be time-variant.
%
% The fluctuations are relatively slow, and therefore a filter is constructed to
% apply the log-amplitude fluctuations as this also allows to take into account the
% frequency-dependency of the fluctuations as shown in equation
% \ref{eq:variances}. This filter is obtained by taking the Inverse Discrete
% Fourier Transform (IDFT) of $\exp(\chi)$. From equation \ref{eq:variances} it
% also follows that the phase-fluctuations have a linear-phase. Therefore, the
% phase fluctuations are converted to a propagation delay fluctuation and applied
% with a Variable Delay Line.
%
% The variance of the dynamic refractive-index $\sigma_{\mu}^2$ was \SI{3e-5}{}
% and the correlation length $L$ \SI{10}{\meter}. The sequences of fluctuations were
% initially computed at \SI{100}{\hertz}. The autocorrelation filter had a length
% of 8192 samples. The time-variant filter for the amplitude modulations had 512 samples
% and hop size was 128 samples. The impulse response transitions were not smoothed.
%
% Figure \ref{fig:implementation:propagation:turbulence} shows a spectrogram of the immission when atmospheric attenuation is included.
%
% \begin{figure}[H]
%   \centering
%   \includegraphics[]{../figures/generated/propagation/turbulence}
%   \caption{Propagation turbulence signal.}
%   \label{fig:implementation:propagation:turbulence}
% \end{figure}

\newpage
\subsection{Ground reflection} % TODO refer to theory. Extend?
While the image source method (ISM) was implemented with image
receivers instead of image sources, only one additional path,
a ground reflection, was considered. Therefore, the implementation of the ISM
will not be treated any further.

The ground reflection is considered as a second propagation path using a mirror receiver.
% The implementation supports direction-dependent emission signals, but because we had only one emission...
For the ground reflected path the same emission signal was used as for the
direct path, thereby ignoring directivity of the sources. Why the directivity of
the source is ignored will become clear in the following sections.

A filter was included to take into account the transfer funtion of the ground.
For the ground reflection the plane wave reflection coefficient was used (eq. \eqref{eq:theory:sound:reflection:plane}) and the impedance $Z$ was
calculated using Delany and Bazley's one-parameter model (eq. \eqref{eq:theory:sound:impedance:db}).

Because aircraft are mostly overhead and at larger distances,
using the plane wave reflection coefficient should typically be sufficient.
However, when simulating noise sources at low elevation angles, the plane-wave
assumption is no longer valid and must be replaced by a model that takes into
account the reflection of spherical waves from a ground surface of finite
impedance \cite{Tuttle2014}. % TODO rewrite. Maybe move to theory

The scenarios considered were landings. Because the area around the airport
consists mostly of grass a flow resistivity of
\SI{2e5}{\pascal\second\per\meter\squared} was chosen. The filter length was
4096 samples and the hop size 256 samples.

Figure \ref{fig:implementation:propagation:reflection} shows a spectrogram of
the immission when a ground reflection was added. A hard ground was considered for this figure.
The effect of a soft ground was already demonstrated in Figure \ref{fig:theory:sound:reflection:ground}.

\begin{figure}[H]
  \centering
  \includegraphics[]{../figures/generated/propagation/reflection}
  \caption{Spectrogram of the immission when a ground reflection is present. The Lloyd's mirror effect is clearly visible.}
  \label{fig:implementation:propagation:reflection}
\end{figure} % TODO text


%
%
%
%
% TODO: OLD PART
%
%
% Figure \ref{fig:propagation_model_propagation_model} represents a block
% diagram of the steps that are taken. The steps are taken for every
% source-(mirror)receiver combination.
%
%
% \begin{figure}[H]
%         \centering
%         \includegraphics[height=0.4\textheight]{../figures/propagation_model}
%         \caption{Block diagram of propagation model.}
%         \label{fig:propagation_model_propagation_model}
% \end{figure}
%
%
%
% \subsection{Attenuation due to spreading}
%
% Geometrical spreading
% \begin{equation}
%  p_t = p_0 \frac{r_0}{r_t}
% \end{equation}
%
% \newpage
% \subsection{Time delay and Doppler shift}
%
% The limited speed of sound causes a delay between receiving and emitting a signal.
% A common method to include this delay in a real-time implementation is to use a variable-delay line.   % Needs reference
% % This implementation is not real-time, resulting in more available options.
%
% % The time delay can be simulated using a variable delay line.
%
% Since the signal is discrete and the delay is generally not an integer multiple of the sample time, an interpolation scheme is required.
%
% \subsubsection{Linear interpolation}
% Initially, linear interpolation was used.
% For a given sound travel time $\Delta t(t)$ from source to receiver, the index
% $k_{r}^{'}$ is given as
% \begin{equation}
%  k_{r}^{'} = k_e + \Delta t (t) \cdot f_s
% \end{equation}
% where $f_s$ is the sampling frequency.
% ......
%
% Figure \ref{} shows an example where linear interpolation was used to apply a Doppler shift.
%
% \missingfigure{Spectrogram of a Doppler shifted tonal component. The Doppler shift was applied by resampling the signal and using linear interpolation.}
%
% The Doppler shifted tonal component is clearly visible, however, strong artefacts are also visible and indeed audible.
% In practice, these artifacts are often masked by noise components but this however cannot be guaranteed.
%
% There are methods to reduce these artefacts. For example, upsampling the signal before resampling decreases the prominence of the artifacts.
% Another possibility is a different interpolation scheme.
%
% \subsubsection{Lanczos interpolation}
% Theoretically, the optimal reconstruction filter is a sinc filter. In practice
% only approximations of this filter can be used, and these approximations are
% generally achieved by windowing and truncating the sinc function. One of these
% approximations is the Lanczos filter or kernel, which is the sinc function
% windowed by another sinc function.
%
% The Lanczos kernel is given by
% \begin{equation}
%  L(z) = \begin{cases}
%          \textrm{sinc}(z) \textrm{sinc}(z/a), & \textrm{if} -a < z < a \\
%          0, & \text{otherwise}
%         \end{cases}
% \end{equation}
% where $a$ is the size of the kernel. Consider now a signal with samples $s_i$
% for integer values of $i$ where sample $s_i$ corresponds to the sample at $t=i/f_s$.
% The value at retarded time $t'$ is then given by
% \begin{equation}
%  S(x) = \sum_{\floor{x} - a + 1}^{\floor{x} + a} s_i L(x-i)
% \end{equation}
% where $x$ is the sample at retarded time $t'$
% \begin{equation}
%  x = -t' + i
% \end{equation}
% The frequency shift depends on the change of propagation delay. Therefore, when source and receiver are relatively close to one another, the method is most sensitive to uncertainties in source position and speed of sound.
%
% \missingfigure{Spectrogram of a Doppler shifted tonal component. In this case the Doppler shift was applied by resampling the signal and using Lanczos interpolation. The artefacts are clearly less pronounced as those present in figure \ref{}}
%
% \missingfigure{Spectrogram of auralisation}
%
% \newpage
% \subsection{Atmospheric attenuation}
% The atmospheric attenuation coefficient $\alpha$ in dB/km is calculated according to ISO 9613-1 \cite{ISO9613-1} as presented in \ref{sec:theory_sound_atmospheric_attenuation}.
%
% A single-sided spectrum is calculated as
% % Equation \ref{} is sampled to obtain an attenuation spectrum in dB/m and then multiplied with the current distance to obtain the actual attenuation.
% \begin{equation}
%  10.0^{- d \alpha(f) / 20}
% \end{equation}
% The spectrum is real and even, and therefore the impulse response is real and even as well. A non-causal zero-phase filter was designed.
%
%
%
% The attenuation spectrum is converted to an impulse response using the IFFT.
% Because the attenuation is distance-dependent and the distance changes over time, the impulse response needs to be updated.
%
% The attenuation is applied by performing a convolution between the impulse responses and the input signal.
% The convolution is done by multiplying a matrix consisting of impulse responses with a vector representing the input signal.
% The impulse responses are stored in a sparse matrix to reduce memory consumption.
% Nevertheless, because of computational limitations, the impulse response describing atmospheric attenuation are determined for an $N$ amount unique distances.
%
% Since atmospheric absorption is frequency-dependent, care should be taken that
% one corrects for the Doppler shift first. Since a moving medium (due to wind) is not
% included in the propagation model, the time delay between source and receiver
% can be calculated directly, and the atmospheric absorption applied thereafter.
%
% \missingfigure{Spectrogram of auralisation}
%
% %
% % The attenuation coefficient for
% % pure tones $\alpha$ is calculated according to ISO 9613-1 \cite{ISO9613-1} in
% % the frequency domain. This attenuation, corresponding to a (single-sided) power spectrum in decibels, is converted to a double-sided amplitude spectrum
% % using
% % \begin{alignat}{2}
% %  a_{\alpha}[k] &= 10^{+ d \alpha}, && \quad 0 \leq k \leq N/2 \\
% %  a_{\alpha}[-k] &= a_{\alpha}[k], && \quad 1 \leq k \leq N/2-1
% % \end{alignat}
% % where $M$ is the amount of desired filter taps, $k$ the block index of discrete frequency $f_k$ and $d$ the source-receiver distance.
% % Note also the plus sign; in case of an auralization this sign would be negative but now we're interested in an amplification.
% % An impulse response is obtained by taking the IFFT of $M$ blocks. The small imaginary parts are discarded by taking the real part.
% % The filter is then made causal by rotating the impulse response with $M/2$ samples. A rectangular window was used.
% %
% % The attenuation is range-dependent and because of the aircraft movement the range varies with time. Within one recording there is a
% % maximum and minimum range. An $N$ amount of impulse responses are determined for
% % an $N$ amount of equispaced ranges.
% %
% % The attenuation is applied by performing a convolution between the impulse
% % responses and the input signal. The convolution is done in a rather naive way by multiplying a
% % Toeplitz matrix consisting of impulse responses with a vector representing the input
% % signal. The impulse responses are stored as a sparse array to reduce memory
% % consumption. No further operations were done regarding the impulse response transitions.
%
% \newpage
% \subsection{Atmospheric turbulence}
%
% - per octave band
%
% \missingfigure{Spectrogram of auralisation}
%
% \newpage
% \subsection{Reflections and shielding}
%
% In an urban environment reflections and shielding play a role.
%
% Hard surfaces except possibly the ground.
% Surfaces relatively smooth considering wavelength. Specular reflections are assumed.
%
% Considering the large distances compared to the wavelengths the waves can likely
% be assumed to be plane and a plane-wave reflection coefficient can be used. Then
% again, the increased computational effort of using a spherical reflection
% coefficient is negligible.
%
% \missingfigure{Spectrogram of auralisation}
%
% Using the image source method mirror receivers are determined.
% Their strength and effectiveness is determined every $N$ samples and the mirror receivers are sorted on strength.
% The strongest sources are selected.
%
% \newpage
% \subsection{Inverse propagation model}\label{sec:propagation_model_reverse_propagation_model}
% Based on this propagation model an inverse propagation model was developed.
% The inverse propagation model is used to calculate back to the source in order to determine emission characteristics.
%
% Figure \ref{fig:propagation_model_reverse_propagation_model} shows a block
% diagram of the steps taken in the reverse propagation model.
%
% A likely significant error that is being made is that the ground effect is
% still included.
%
%
% \begin{figure}[H]
%         \centering
%         \includegraphics[width=0.4\textwidth]{../figures/reverse_propagation_model}
%         \caption{Block diagram of reverse propagation model. A sound recording
% is used in conjunction with detailed flight path information. Spherical
% spreading, atmospheric absorption, and the Doppler shift are undone, resulting
% in a source signal.}
%         \label{fig:propagation_model_reverse_propagation_model}
% \end{figure}
