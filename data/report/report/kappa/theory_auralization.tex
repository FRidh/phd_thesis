\section{Auralisation and sound reproduction}

\subsection{Synthesis techniques}
Synthesis techniques have been categorised in different ways. In \cite{JuliusO.Smith} digital synthesis techniques are arranged into four categories:
\begin{itemize}
 \item processed and sampled records
 \item spectral models
 \item physical models
 \item abstract algorithms
\end{itemize}

Misra and Cook produced an overview of synthesis methods for sound designers and 
composers.
According to them the following methods can be used for synthesis of textures 
and soundscapes:
\begin{itemize}
 \item concatenative / granular synthesis
 \item Linear Predictivite Coding (LPC)
 \item stochastic methods
 \item wavelet-based methods
\end{itemize}


\subsection{Parametric sound representation}
Yercoe et al. present an overview of parametric sound representations \cite{Vercoe1998}.

\subsection{Reproduction}


\newpage
\subsection{Software}\label{sec:theory:auralisation:software}
Eventually software is needed for synthesis and auralisation. What software or
tools to use depends on the requirements. What is the purpose of the software?
Who is going to use it? Under what conditions is it going to be used? Is it
going to be used online or only offline? How should it perform? How usable and
learnable should it be? And is it supposed to interoperate with other tooling?
Questions like these may determine what tooling to choose. What follows is a
brief, though incomplete, overview of software that may be used for (real-time)
synthesis and auralisation.

Max is a proprietary visual programming language for multimedia originally
written in the 1980s \cite{Max2017}. With the Max Signal Processing (MSP) add-on
package real-time signal processing is possible. Pure Data is written by the
original author of Max and is a free and open-source alternative
\cite{PureData2017}. Both are dataflow programming languages where programs are
made by linking objects or blocks together into a block diagram.

SuperCollider is an environment and programming language for real-time audio
synthesis \cite{SuperCollider2017} . The SuperCollider language
(\mintinline{sh}{sclang}) combines features from object-oriented languages with
those from functional languages. SuperCollider uses a server
(\mintinline{sh}{scsynth}) that can be controlled by external programs. External
programs that interpreter the \mintinline{sh}{sclang} language control what is
being synthesised on the \mintinline{sh}{scsynth} server through Open Sound
Control (OSC) , which is a protocol for connecting sound synthesisers, computers
and other hardware that can be used for performances.

Simulink is a proprietary graphical programming environment for modeling and
simulating dynamic systems \cite{Simulink2017}. Like with the previous
languages, programs are composed through the graphical user interface and
consist of a set of blocks forming a block diagram. While not specifically
targeting sound synthesis, the provided blocks, its integration with MATLAB,
and its possibility to deploy programs on external DSPs, can make Simulink a
convenient and interesting environment for synthesis and auralisation.

% Several domain-specific programming languages for signal processing exist.

FAUST is a functional programming language for real-time signal processing and
sound synthesis \cite{Faust2017}. The FAUST compiler translates FAUST code into
C++, no runtime is needed. Digital signals are modeled as discrete functions of
time, signal processors are functions that operate on the signals, and
composition operators are used to combine signal processing functions.
This allows one to take a block diagram approach when writing programs, and
eventually allows a C++ compiler to heavily optimise the generated C++ code.

Finally, libraries for signal processing are available natively or as bindings
for many languages.




%
% While any Turing-complete language may be used to
%
% Whereas
%
%
% A major part of any synthesis or auralisation tool would
% deal with signal processing.
%
% For synthesis and auralisation different software or tooling can be used. A
% significant part of the tooling would deal with the signal processing steps
% involved.
%
%
%
%
% What tools or software to use depends on the set requirements. What
% follows is a brief overview of software that could be used.
%
%
%
%
% A significant part of auralisation software is deals with the signal processing steps involved.
%
%
% Simulink
%



% \subsection{Ambisonics}
%
%
% Increasing the order of ambisonic reproduction allows 3D sound fields to be reconstructed with improved resolution.
%
%
% \subsection{Correct over-head information}
% For the reproduction of the auralisation a system has to be set up. One issue to
% cope with is correct over-head information. There exists a localisation error
% when using an Ambisonics system instead of real sources.
%
% Power et al. conducted subjective localisation tests to determine the spatial accuracy of Ambisonic systems when simulating elevated sources \cite{Power2013}.
% They report the results of subjective localisation tests for virtual sources placed in the vertical plane at various elevations and azimuths, for first, second and third-order ambisonic reproduction over a 16-loudspeaker system.
% An error in elevation angle of 16 to 33 degrees for first and
% second order Ambisonics systems and 10 to 22 degrees for third order
% systems was found \cite{Power2013}.
%
%
% \cite{Power2013}
%
% \subsection{Perception of distance}
%
% \cite{Kearney2012}
