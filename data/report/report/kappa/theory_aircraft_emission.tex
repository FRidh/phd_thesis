\section{Aircraft noise sources}

% In previous sthe previous section we discussed the theory of sound generation and propagation.
% In this section we will have a closer look at the different noise sources on an
% airplane and how those can be modeled.
%
% \subsection{Noise sources}

There are several noise sources on an airplane and their level of contribution
depends on the operating conditions as well as airplane specifics, like for
example what type of engines are used. The main noise sources on a jet airplane
are the jet engine and aerodynamic noise from different parts of the airframe
\cite{Zaporozhets2011}. Noise contributions from the jet engine can be further
divided into subsources. Relevant subsources are the fan, combustor, turbine and
finally the jet of the exhaust flow that is projected out of the nozzle. For
noise reduction acoustic liners are typically present. Some aircraft may also
have sawtooth patterns on the nozzle known as chevrons.

\subsubsection{Engine types}
The engine is the component generating power and providing thrust. Several kinds
of engines are used in airplanes.

A turbojet engine consists of a combustion turbine with a propelling nozzle. Air
is taken by an inlet and compressed by a compressor that is driven by the
turbine. The compressed air is heated in the combustion chamber, then passes
through the turbine expanding and driving it, and finally expands further in the
nozzle where it is accelerated to provide thrust. Because of low fuel-efficiency
most aircraft use a different engine nowadays.

Smaller aircraft typically use a turboprop engine. A turboprop engine is similar
to a turbojet engine. Thrust is generated not by the outgoing flow, but by a
propeller which is driven by the turbine.

The turbofan engine is mostly used by larger aircraft and is quite similar to
both turbojet and turboprop engines. A fan is placed in front of the inlet.
Whereas with a turbojet all the air taken in by the inlet passes through the
turbine, with a turbofan some of the air taken in by the fan bypasses the
turbine. Thrust is provided not only through the nozzle but also by the fan. The
bypass ratio is defined as the ratio between the mass-flow rate of the bypass stream
to the mass-flow rate entering the turbine. Commercial airliners typically have
\say{high-bypass} turbofan engines that provide more fan-thrust relative to
jet-thrust. Turbofan engines proved for commercial airliners to not only be more
fuel-efficient, but also reduced their noise emission.
% TODO citation?

\subsubsection{Fan}
In a turbofan engine a fan is used for generating thrust. The blades rotate at
high angular frequency, with the tips of the blades moving at sub- or supersonic
Mach numbers. Unsteady flow due to inflow turbulence and flow separation from
the blades causes broadband noise. Steady and unsteady forces acting on the flow
cause tonal components at the blade passing frequency of the fan and harmonics.
% The steady force of the rotating fan acting on the flow as well as unsteady forces cause tonal components.

When the tips of the blades move at supersonic Mach numbers, shocks are created
resulting in multiple pure tones. A shock is created by each blade. Due to
blade-to-blade differences the waves coalesce. Tonal components will therefore
exist at the angular frequency of the engine shaft. These tonal components are
called \say{Buzz-Saw} noise and its time-domain signal resembles a sawtooth.
The \say{Buzz-Saw} noise is common during take-off.

% TODO citations
% A classic theoretical model for noise from propellers is the Gutin model, where sound is
% generated by a time-varying force field created by the rotating propeller.
% Ffowcs Williams and Hawkings later provided a more general theory for rotating
% propellers. The Headmann model is a well known model for predicting fan noise in case
% of jet engines.

\subsubsection{Combustor}
To drive a turbine compressed air is heated in the combustion chamber. Heat is
provided by burning fuel. Noise is created due to the combustion process as well
as due to turbulent airflows. Low-frequent noise can propagate through the
turbine and exhaust, and this type of noise is known as core noise \cite{Zaporozhets2011}.


\subsubsection{Airframe}
Airframe noise is noise that is generated by aerodynamic flow around the
airframe and is an important noise source at low engine power settings.
Aerodynamic noise can be generated on the tail, wing, flaps, slats and landing gear.
Furthermore, airframe noise includes the airfoil-tip vortex \cite{Zaporozhets2011}.


\subsubsection{Jet noise}
Jet noise was briefly discussed in section \ref{sec:theory:generation}. The
exhaust flow is projected in the ambient air and turbulent mixing of the two
fluids causes jet noise.

% , where $U$ is the jet
% outflow velocity.



% \begin{equation}
%   \overline{W}_{jet} \propto \frac{\rho_{jet} U^8 D^2}{ \rho_0 c_0^5}
% \end{equation}
% where $D$ is the jet diameter and $\rho_{jet}$ the density of the jet.

% \subsection{Noise prediction models}


% \subsection{Emission synthesis}


% % \subsection{Aircraft spectra}\label{sec:theory_aircraft_emission_aircraft_spectra}
% %
% % \subsection{Aircraft emission models}\label{sec:theory_aircraft_emission_models}
%
% \subsubsection{Jet noise - Stone}
%
% \subsubsection{Fan noise - Heidmann}
%
% \subsubsection{Combustion noise}
%
% \subsubsection{Airframe noise - Fink}
%
%
%
% \subsection{Operations}
%
% \subsection{Emission synthesis}
