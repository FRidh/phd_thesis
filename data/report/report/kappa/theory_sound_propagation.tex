\section{Propagation of sound}

\subsection{Doppler shift}
A commonly known effect related to movement is the Doppler shift.
Relative motion between source, receiver and medium can cause
Doppler shifts. If we consider a moving harmonic point source, $S(t) = \hat{S}_0
\exp{\left(j\omega_0 t\right)} $, a homogeneous fluid at rest, and a non-moving receiver, then the sound
field is given by
\begin{equation}
  p'(\vect{x},t) = \sum_{t_e} \frac{\hat{S}_0 \exp{\left(j \omega_0 t_e\right)}}{4\pi r \left(t_e\right) \abs{1-M_r (t_e)}}
\end{equation}
and represents a summation of the sound pressure contributions at all emision
times $t_e$. The Mach number $M_r = \vect{M} \cdot \vect{e}_r = M \cos{\theta}$ is
the projection of the Mach number of the source in the direction towards the
receiver. The Mach number of the source is $\vect{M}=\vect{v}/c$ and
$\vect{e}_r=\vect{r}/r$ is the unit vector pointing towards the receiver.
The instantaneous angular frequency is then given by
\begin{equation}
  \omega = \frac{\omega_0}{1 - M_r(t_e)}
\end{equation}
% with $M_r(t_e) = M(t_e) \cos{\theta(t_e)}$.

% \chapter{Sound propagation}


\subsection{Atmospheric attenuation}\label{sec:theory_sound_atmospheric_attenuation}
As sound waves travel through an atmosphere the waves are attenuated through
atmospheric attenuation. There are two effects causing attenuation. The first
effect causing attenuation is classical heat conduction and shear viscosity.
These effects were considered insignificant when deriving the wave equation. The
second effect is on a molecular scale and is due to an energy exchange between
translation and rotational or vibrational modes. Molecules collide and are
excited when a wave passes resulting in energy transfer. The energy transfer is
a function of the relaxation time of the molecules. Because the molecules need a
certain time to reach equilibrium the heat capacity of the medium is
time-dependent thus resulting in an irreversible process causing losses.

A model for atmospheric attenuation is given in Part 1 of ISO 9613-1:1993 \cite{ISO9613-1}.
The attenuation coefficient $\alpha$, in dB/m, is given by
\begin{align}\label{eq:theory:sound:atmospheric-attenuation}
 \alpha &= 8.686 f^2 \Biggl( \left[ 1.84 \times 10^{-11} \left(\frac{p_r}{p_a}\right)^{-1} \left(\frac{T}{T_0}\right)^{1/2} \right] + \left(\frac{T}{T_0}\right)^{-5/2} \nonumber \\
 &\times \Biggl\{ 0.01275 \left[ \exp{\frac{-2239.1}{T}} \right]  \left[f_{r,O} + \frac{f^2}{f_{r,O}} \right]^{-1} \nonumber \\
 &+ 0.1068 \left[ \exp{\frac{-3352.0}{T}} \right] \left[ f_{r,N} + \frac{f^2}{f_{r,N}} \right]^{-1} \Biggr\} \Biggr)
\end{align}
and is a function of the ambient temperature in kelvin $T$, the reference
temperature $T_0=293.15$ K, the ambient pressure $p_a$ in kilopascal, the
reference pressure $p_r=101.325$ kPa and the relaxation frequencies for oxygen
$f_{r,O}$ and nitrogen $f_{r,N}$.
The relaxation frequency of oxygen is given by
\begin{equation}
 f_{r,O} = \frac{p_a}{p_r} \left( 24 + 4.04 \cdot 10^4 h \frac{0.02 + h}{0.391 + h}  \right)
\end{equation}
and the relaxation frequency of nitrogen by
\begin{equation}
 f_{r,N} = \frac{p_a}{p_r} \left( \frac{T}{T_0} \right)^{-1/2} \cdot \left( 9 + 280 h \exp{\left\{ -4.170 \left[ \left(\frac{T}{T_0} \right)^{-1/3} -1 \right] \right\} } \right)
\end{equation}
Both depend on the molar concentration of water vapour $h$, given by
\begin{equation}
 h = h_r  \frac{p_{sat}}{p_a}
\end{equation}
The molar concentration of water vapour is a function of the saturation pressure
\begin{equation}
 p_{sat} = 10^C \cdot p_r
\end{equation}
where
\begin{equation}
 C = -6.8346 \cdot \left( \frac{T_{01}}{T} \right)^{1.261}  + 4.6151
\end{equation}
and where $T_{01}$ is the triple-point isotherm temperature of 273.16 K.

% % TODO what to do with this
% The standard furthermore mentions the following expression for the speed of sound
% \begin{equation}
% c = 343.2 \left( \frac{T}{T_0} \right)
% \end{equation}

\begin{figure}[H]
        \centering
        \includegraphics[]{../figures/generated/sound/attenuation}
        \caption{Atmospheric attenuation as function of frequency for a standard atmosphere.}
        \label{fig:theory:sound:attenuation}
\end{figure}

\newpage
\subsection{Reflections}
Sound hitting a boundary between materials of different impedance may be
transmitted, reflected or absorbed. When sound is transmitted the incident wave
is partially refracted. Sound reflected by a smooth surface results in a
specular reflection, and sound reflected on a rough surface results in a diffuse
reflection. In outdoor situations sound is typically reflected by the ground
surface and other obstacles like buildings. The sound pressure at a receiver is
then the sum of the direct contribution as well as indirect contributions from
reflections. Due to this superposition strong interference effects may occur.

\subsubsection*{Impedance}
An impedance is a complex ratio between two interdependent dynamic quantities
and describes the resistance to a direct flow (resistance) and alternating flow
(reactance). The specific acoustical impedance of a material is the ratio of the
sound pressure to the particle velocity normal to the surface $Z_{s} =
p/\vect{u}$. The specific acoustical impedance of a material or surface
normalised by the impedance of air $Z_{s, air} = \rho c$ is called the
normalised specific acoustic impedance. This impedance is typically used to
characterise surfaces, and is what impedance in the following text refers to.

Multiple models are available for the prediction of the impedance of a surface.
Attenborough et. al. made a comparison of impedance models and recommends the
Delany and Bazley one-parameter model for predicting outdoor ground impedance
\cite{Attenborough2011a}. The empirical one-parameter model by Delaney and
Bazley is widely used and depends on the flow resistivity of the material
$\sigma$ and the frequency $f$
\begin{equation}\label{eq:theory:sound:impedance:db}
 Z = 1 + 9.08 \left( \frac{1000f}{\sigma}\right)^{-0.75} - 11.9 j \left( \frac{1000f}{\sigma}\right)^{-0.73}
\end{equation}
Flow resistivity describes how difficult it is for air to flow through a
surface. Values for surfaces can vary significantly, with average values ranging
from \SI{29}{\kilo\pascal\second\per\square\meter} for snow to
\SI{200000}{\kilo\pascal\second\per\square\meter} for painted concrete. Grass
has an average flow resistivity of
\SI{200}{\kilo\pascal\second\per\square\meter} \cite{Crocker1997}.
Furthermore, a surface is called locally reacting if at a certain point the
particle velocity $\vect{u}$ depends only on the sound pressure $p$ at that
point. In case it does not, the surface has an extended reaction.
% Figure \ref{fig:theory:sound:impedance} shows the impedance of grass according to the Delaney and Bazley model.
% % TODO drop or expand?
%
% \begin{figure}
%         \centering
%         \includegraphics[]{../figures/generated/sound/impedance}
%         \caption{Impedance for grass with a flow resistivity of \SI{200}{\kilo\pascal\second\per\square\meter} according to Delany and Bazley one-parameter model.}
%         \label{fig:theory:sound:impedance}
% \end{figure}


%
% Another example of a model is the 2-parameter model by Attenborough. In this model, the impedance is given by
% \begin{equation}\label{eq:theory:sound:impedance:att}
%  Z = \frac{\left( 1-j\right) \sqrt{\sigma/f}}{\sqrt{\pi \gamma_0 \rho_0}} - \frac{jc\alpha}{8 \pi \gamma_0 f}
% \end{equation}
% and depends on the speed of sound in air $c_0$, the density of air $\rho_0$, $\alpha$, $\gamma_0$ and again the flow resistivity $\sigma$ and frequency $f$.

%
% Table \ref{tab:theory:sound:impedance:flow-resistivity} lists co
%
%
% \begin{tabular}{l c}
% Team              & P & W & D & L & F  & A & Pts \\
% \hline
% Manchester United & 6 & 4 & 0 & 2 & 10 & 5 & 12  \\
% Celtic            & 6 & 3 & 0 & 3 &  8 & 9 &  9  \\
% Benfica           & 6 & 2 & 1 & 3 &  7 & 8 &  7  \\
% FC Copenhagen     & 6 & 2 & 1 & 3 &  5 & 8 &  7  \\
% \end{tabular}


% \newpage
\subsubsection*{Reflection coefficient}
The pressure reflection coefficient describes the ratio between the sound
pressure of an incident wave $p_i$ and the pressure of the reflected wave $p_r$
and considers a wave incident on an infinite plane that is locally
reacting and has impedance $Z$
\begin{equation}
  R = \frac{p_r}{p_i}
\end{equation}
The plane wave reflection coefficient gives the ratio of incident and reflected pressure assuming the incident wave is plane
\begin{equation}\label{eq:theory:sound:reflection:plane}
  R = \frac{Z\cos{\theta}-1}{Z\cos{\theta}+1}
\end{equation}
and is a function of the angle of incidence $\theta$ of the incident wave.
If the source is relatively close to the reflecting surface,
then typically the wave front is not plane. In such case a spherical wave
reflection factor is used that considers an incident spherical wave.

The impedance and reflection coefficient models together with their parameters
determine in computations how much sound is reflected and how much is
transmitted or absorbed. Because the reflection coefficient is complex-valued
the reflected contribution may have not only a different magnitude than the
direct contribution, but also an additional frequency-dependent phase shift or
propagation delay. Surfaces that attenuate and phase shift the reflected wave
are called acoustically soft surfaces. In case the normalised impedance
approaches infinity, neither attenuation nor phase shift occur, and the surface
is considered acoustically hard.
% Figure \ref{fig:theory:sound:reflection} shows
% both the absolute value and the phase angle of the plane wave reflection angle
% as function of frequency and angle of incidence.

% Another commonly used reflection coefficient is the spherical wave reflection coefficient.

% The spherical reflection coefficient is given by
% \begin{equation}\label{eq:theory:sound:reflection:spherical}
%  Q = R \left(1 - R \right) F
% \end{equation}
% with
% \begin{equation}
%  F = 1 - j \sqrt{ \pi} w e^{-w^2} \mathrm{erfc} \left( j w \right)
% \end{equation}
% and
% \begin{equation}
%  w = \sqrt{-j k r  \left( 1 + \frac{1}{Z} \cos{\theta} - \sqrt{1 - \left( \frac{1}{Z} \right)^2} \sin{\theta} \right) }
% \end{equation}

% \begin{figure}
% %     \centering
%     \begin{subfigure}{\textwidth}
%         \includegraphics{../figures/generated/sound/reflection-abs}
%         \caption{Absolute value of the reflection coefficient.}
%     \end{subfigure}
%     ~
%     \begin{subfigure}{\textwidth}
%         \includegraphics{../figures/generated/sound/reflection-angle}
%         \caption{Phase angle of the reflection coefficient.}
%     \end{subfigure}
%     \caption{Absolute value and phase angle of the plane wave reflection angle as function of both frequency and angle of incidence. The impedance was calculated using equation \eqref{eq:theory:sound:impedance:db} and for the flow resistivity the average value for grass was chosen.}
%     \label{fig:theory:sound:reflection}
% \end{figure}

\subsubsection*{Ground effect}
In outdoor situations the ground is typically providing the second-largest
contribution and is important when considering an elevated source. As mentioned
before, impedances of surfaces can vary significantly, and this is especially
the case for the ground surface. Spectrograms of recordings of aircraft often
show a distinct interference pattern called the Lloyd's mirror effect. Figure
\ref{fig:theory:sound:reflection:ground} shows a synthesised example of the
Lloyd's mirror effect as caused by an elevated moving source.

\begin{figure}
%     \centering
    \begin{subfigure}{\textwidth}
        \includegraphics{../figures/generated/sound-mirror-effect/hard}
        \caption{Hard surface.}
    \end{subfigure}
    ~
    \begin{subfigure}{\textwidth}
        \includegraphics{../figures/generated/sound-mirror-effect/soft}
        \caption{Soft surface.}
    \end{subfigure}
    \caption{Superposition of the direct contribution and ground-reflected contribution results in an interference pattern known as the Lloyd's mirror effect.}
    \label{fig:theory:sound:reflection:ground}
\end{figure}
% TODO interference pattern is shifted}

% A prominent reflection is the reflection with the ground. When the
% reflecting surface is sufficiently hard the Lloyd's
% mirror effect can be heard.


% \begin{figure}[H]
%         \centering
%         \includegraphics[width=0.9\textwidth]{../figures/ipynb/theory_reflections_mirror_effect/figure1}
%         \caption{Spectrogram of a white noise source flying at a height of 100 meters over an acoustically hard surface. The spectrogram clearly shows the Lloyd's mirorr effect.}
%         \label{fig:theory_reflections_mirror_effect}
% \end{figure}

% \missingfigure{SPL relative to free field comparison}


\subsubsection*{Multiple reflections and shielding} % TODO don't talk about shielding?
In urban environments there are more reflecting surfaces besides the ground.
Buildings can reflect sound and shield as well. In courtyards or street canyons
multiple reflections and strong rerverberance may occur.

Sound can be considered as a ray if the wavelength of the sound is much smaller
than the characteristic lengths of the objects or geometries. % TODO: move sentence

Assuming sound can be described as a ray, a raytracer could be used to determine
the immission at a receiver position, taking into account refractions and
reflections. Raytracers are commonly used in room acoustics but also in
environmental acoustics for noise prediction.

The Image Source Method (ISM) is a computationally fast algorithm for taking into
account reflections and uses the concept of mirror sources
\cite{Allen1979,Mechel2013}. Mirror sources are found by mirroring the original
source with respect to the reflective surface that is considered. Higher-order
mirror sources can be found by repeating the process, mirroring the previous
order mirror source with respect to another surface. Care should be taken to
determine whether there is in fact line-of-sight between a mirror-source and the
receiver.

When neither source nor receiver move, the image sources have to be determined
only once, and their validity, that is, whether there is line-of-sight between
image source and receiver, needs to be checked only once as well. If instead the
receiver moves, line-of-sight will have to be checked as function of time. When
both move, both the mirror source search as well as the line-of-sight validation
has to be performed as function of time. An interesting question is how to
connect image sources at one instance in time to the image sources the next
instance, considering sources can appear and disappear over time. If only the
source moves, it may be more interesting to consider mirror receivers instead of
miror sources.

% In urban environments multiple reflections generally occur.


% \subsection{Descriptors}


\subsection{Atmospheric turbulence}\label{theory:sound-propagation:atmospheric-turbulence}

% TODO: rewrite paragraph from paper
In an outdoor situation, spatial and temporal variations in temperature and wind
velocity cause small changes in the refractive index. As waves pass through the
atmosphere, the index-of-refraction variations in effect cause scintillations,
i.e. fluctuations or modulations in the received intensity of the wave.
Scintillations affect both sound and electromagnetic waves. They are a major
limitation for astronomical observations using Earth-based telescopes and also
reduce performance of wireless communication systems. Scintillations can also be
noticed when hearing sound emitted by a source at a large distance, like for
example by an aircraft or distant windturbine \cite{Heutschi2014}.
Shown is in this section how weak fluctuations
cause sound pressure perturbations. Weak turbulence fluctuations are
non-linearities but they are assumed to be small.


\subsubsection*{Turbulent flow}
In a laminar flow, fluid motion is smooth. Fluid \say{particles} move parallel
to eachother in the flow direction and viscous forces dominate. In a turbulent
flow, intertial forces dominate, fluid motion is chaotic and fluid particles no
longer follow a straight path. Instead, they may loop, forming instabilities
such as eddies and vortices. The Reynolds number $\mathrm{Re}$ may be used to
predict the transition from laminar to turbulent flow and is the dimensionless
ratio between inertial and viscous forces.
% \begin{equation}
%   \mathrm{Re} = \frac{\rho \nu L}{\mu}
% \end{equation}
% where $\rho$ is the fluid density, $\nu$ the characteristic velocity of the
% fluid, $L$ a characteristic dimension, and $\mu$ the dynamic viscosity of the
% fluid.

% \subsubsection*{Eddies}
The onset of turbulence and formation of eddies can be explained by considering
a particle that moves faster than its neighbouring particles. As it encounters
other particles it is forced to deviate from its straight course. The term eddy
is used not only for loops or swirls. The region wherein velocity fluctuations
exist is also called an eddy. Furthermore, an eddy corresponds not only to a velocity
fluctuation, but also to a temperature fluctuation.
The size of the eddies is limited by some characteristic dimension. For example,
when consider a flow through a pipe, eddies are smalles than the pipe diameter.

% \subsubsection*{Frozen turbulence}
If advection\footnote{Advection is the transport of a substance and its properties by bulk motion.} contributed by turbulent circulations is small compared to that of
the mean flow, then the eddies can be considered as \say{frozen} fluctuations
passing through the fluid. The size of the eddy $L$ is then related to a
characteristic time $\tau$ and mean flow velocity of the fluid $u$ by $L = u
\tau$. This assumption is referred to as Taylor's frozen turbulence hypothesis.

\subsubsection*{Atmospheric boundary layer}
In the atmospheric boundary layer the height above the ground is the
characteristic dimension and with increasing height larger eddies occur. These
large eddies break down into smaller eddies forming a broad distribution or
spectrum of eddy sizes and thus turbulent fluctuations. The time scale of
variations or fluctuations in the atmosphere therefore varies wildly, with
variations ranging on scales from seconds or minutes, to hours or days.

% \subsubsection*{Sound propagation}
The temperature and velocity fluctuations cause variations in the
refractive-index of the medium. As sound passes, it is scattered multiple times
along its propagation path resulting in modulations of amplitude and phase.

Atmospheric turbulence is a stochastic process and therefore the resulting modulations are
also random. A random process has a distribution and a power spectral density.
The spectrum of atmospheric turbulence is strongly related to the sizes of the eddies.
Several models are available for describing the spectrum of turbulence.
Furthermore, the shape of the wavefront is relevant when considering sound
propagation through a turbulent atmosphere.


% TODO rewrite
\subsubsection*{Rytov approximation}
Variations in temperature and wind in both position $\mathbf{r}$ and time $t$
cause variations in the refractive index field $n(\mathbf{r},t)$.
Interested in how these variations affected wave propagation, a first-order Rytov approximation is done to obtain expressions for the fluctuating pressure.
The following expressions are based on \cite{Ishimaru1997} and \cite{Jurado-navas2006}, but instead of electromagnetic
waves, sound waves are considered. We consider for now spatial variations only, and
as a starting point we use the following Helmholtz equation
\begin{equation}\label{eq:helmholtz_without_fluctuations}
 \left( \nabla^2 + k^2 n^2(\mathbf{r}) \right) p(\mathbf{r})= 0
\end{equation}
with pressure $p$, wavenumber $k$, and refractive-index field
\begin{equation}\label{eq:refractive_index_field}
 n(\mathbf{r}) = n_0 + n_1(\mathbf{r})
\end{equation}
with mean value $n_0 = E[n(\mathbf{r})] = 1$ and first-order perturbation $n_1(\mathbf{r}) \ll 1$. Merging equations \eqref{eq:helmholtz_without_fluctuations} and \eqref{eq:refractive_index_field} results in
\begin{equation}\label{eq:helmholtz_with_fluctuations}
 \left( \nabla^2 + k^2 (n_0 + n_1(\mathbf{r}))^2 \right) p(\mathbf{r}) = 0
\end{equation}
For weak fluctuations, an approximation to equation \eqref{eq:helmholtz_with_fluctuations} for small $n_1$ is used.
The Rytov solution to equation \eqref{eq:helmholtz_with_fluctuations} is
\begin{equation}
 p = \exp{\left(\psi_0 + \psi_1 + \psi_2 + \dots \right)} = \exp{(\psi)}
\end{equation}
where $\psi_0$ is the complex phase of the unperturbed wave in free space, and $\psi_1$ and $\psi_2$ respectively first-order and second-order complex phase perturbations.

We are interested in the effect of first-order perturbations $n_1$, on the sound pressure,
and therefore write $\psi = \psi_0 + \psi_1$. The refractive index $n$ is written in terms of
an average $\langle n \rangle$ and fluctuation $n_1$, with
\begin{equation}
 \delta n = (1 + n_1)^2 -1 = 2 n_1 + n_1^2
\end{equation}
As derived in \cite{Ishimaru1997}, $\psi_{1}$ satisfies the following integral equation
\begin{equation}
 \psi_{1}(\mathbf{r}) = \frac{1}{p_0(\mathbf{r})} \int_{V'} G(\mathbf{r}-\mathbf{r}') \left[ \nabla \psi_1 \cdot \nabla \psi_1 + k^2 \delta n  \right]  p_0(\mathbf{r}') dV'
\end{equation}
where $G(\mathbf{r}-\mathbf{r}')$ is the free field Green's function. By iteration a series solution can be obtained. For the first iteration we set $\psi_1 = 0$ inside the integral and obtain the first Rytov solution
\begin{equation}
 \psi_{10}(\mathbf{r}) = \frac{k^2}{p_0(\mathbf{r})} \int_{V'} G(\mathbf{r}-\mathbf{r}') \delta n(\mathbf{r}') p_0(\mathbf{r}') dV'
\end{equation}
where $p_0(\mathbf{r})$ is the unperturbed sound pressure field. %, $\mathbf{r}$ a
The sound pressure after the first iteration is then
\begin{equation}
 p (\mathbf{r}) = e^{(\psi_0 + \psi_{10})} = p_0(\mathbf{r}) e^{(\psi_{10})}
\end{equation}

The first-order complex phase perturbation $\psi_{1}$ can be understood as a sum
of waves, generated at various points $\mathbf{r}'$ throughout the scattering
volume $V'$. The strength of each of these waves is proportional to the product
of the unperturbed field term ${p_0}$, and the refractive-index perturbation
$\delta n$ at a point $\mathbf{r}'$ \cite{Jurado-navas2006}.

\subsubsection*{Amplitude and phase fluctuations}
We now want to find expressions for the log-amplitude and phase fluctuations, and will use Rytov's first solution.
We approximate the refractive-index fluctuation as
\begin{equation}
 \delta n = 2 n_1 + n_1^2 \simeq 2 n_1
\end{equation}
and write
\begin{equation}\label{eq:complex_exponential}
 p(\mathbf{r}) = A(\mathbf{r}) e^{jS(\mathbf{r})},  \quad p_0(\mathbf{r}) = A_0(\mathbf{r}) e^{jS_0(\mathbf{r})}
\end{equation}
where $A$ and $S$ are respectively the amplitude and phase of the fluctuating field $p(\mathbf{r})$,
and obtain for the first order perturbations
\begin{equation}\label{eq:chi_and_S}
 \psi_1 ({\mathbf{r}}) = \chi + j S = \log{(A/A_0)} + j (S-S_0)
\end{equation}
In this expression $\chi$ and $S$ represent respectively the log-amplitude fluctuation and phase fluctuation.

By applying the central limit theorem to the first Rytov solution, it follows
that the complex phase follows a normal probability distribution
\cite{Jurado-navas2006}. This is an important result to keep in mind when
generating sequences of fluctuations.

\subsubsection*{Amplitude and phase covariance}
The log-amplitude and phase fluctuations are considered to be the result of a
random temperature fluctuation field. A characteristic of a random function or
field is its correlation function \cite{Tatarskii1971}. The spatial correlation
function of a random field $f(\mathbf{r})$, as function of distance $\mathbf{r}=\mathbf{r}_2-\mathbf{r}_1$
between observation points $\mathbf{r}_1$ and $\mathbf{r}_2$, is defined as
\begin{equation}
 C(\mathbf{r}_1, \mathbf{r}_2) = \langle f(\mathbf{r}_1)  f(\mathbf{r}_2) \rangle
\end{equation}
In a homogeneous and isotropic random field the correlation function $C(r)$
depends only on the distance $r = \lVert \mathbf{r}_2-\mathbf{r}_1 \rVert$
between the observation points and not the path
$\mathbf{r}=\mathbf{r}_2-\mathbf{r}_1$ \cite{Salomons2001}. Note that at this
point, the atmosphere is assumed frozen in time, i.e., variations are only
spatially, not temporal.

We would like to obtain expressions for the covariance functions of the
log-amplitude and phase fluctuations. A specific part of the turbulence spectrum
can be approximated with a Gaussian correlation function
\begin{equation}
 C_{\mu} = \langle \mu(r_1) \mu(r_2) \rangle = \sigma_{\mu}^2 e^{-x^2/L^2}
\end{equation}
where $\sigma_{\mu}^2$ is the variance of the dynamic refractive index,
$x=r_1-r_2$ the distance between two points in space and $L$ the correlation
distance or length \cite{Ishimaru1997}.

% TODO Explain why we go from $x$ to $\rho$ and $d$, i.e., 2 dimensions.
We shall now consider a line-of-sight situation where $d$ is the
distance between the source and a receiver pair along the wave propagation
direction, and $\rho$ the spatial separation of the receivers transverse to the
wave propagation direction.

If the correlation length $L$ is much smaller than the Fresnel zone size of the sound
$\sqrt{\lambda d}$, then the log-amplitude and phase variance scale with
$\sigma_{\chi}^2=\sigma_{S}^2 \sim k^2 d$
\cite{Ishimaru1997} and the variances of the fluctuations are given by \cite{Daigle1983}
\begin{equation}\label{eq:model_daigle}
 \sigma_{\chi}^2 = \sigma_{S}^2 = \frac{\sqrt{\pi}}{2} \sigma_{\mu}^2 k^2 d L
\end{equation}

For spherical waves the covariances of the fluctuations, $B_{\chi}(\rho)$ and
$B_{S}(\rho)$, normalized to their variances, are given by
\begin{equation}\label{eq:theory:propagation:turbulence:correlation}
 \frac{B_{\chi} (\rho)}{\sigma_{\chi}^2} = \frac{B_{S} (\rho)}{\sigma_{S}^2} = \frac{\Phi\left(\rho/L\right)}{\rho/L} = C_{sp}(\rho)
\end{equation}
where
\begin{align}\label{eq:gaussian_correlation}
 \Phi \left(\rho/L \right) &= \int_0^{\rho/L} \exp{\left(-u^2\right)} \mathrm{d} u  \\
 &= \frac{1}{2} \sqrt{\pi} \mathrm{erf}\left( \rho/L \right)
\end{align}
and $\mathrm{erf}$ is the error function.
The covariance functions of the fluctuations $B_{\chi}(\rho)$ and $B_{S}(\rho)$ are thus
\begin{equation}\label{eq:variances}
 B_{\chi} (\rho) = B_{S}(\rho) = \frac{\sqrt{\pi}}{2} \sigma_{\mu}^2 k^2 d L
\frac{\Phi(\rho/L) }{\rho / L}
\end{equation}
% according to Daigle's model.

