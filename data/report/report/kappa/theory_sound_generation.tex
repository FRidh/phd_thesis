\section{Generation of sound}\label{sec:theory:generation}

Classical acoustics provides a method for modelling sound generation due to
vibrating solid boundaries. Aerodynamic sources such as combustion or other
unsteady fluid processes cannot be handled, however. With the development of
jet-powered aircraft a description was needed to model such sources. This
section provides a brief introduction to generation of aerodynamic sound. A more
extensive discussion on generation of aerodynamic sound can be found in
\cite{Rienstra2017}.

\subsection{Aerodynamic sound sources}\label{sec:theory:sound:aerodynamic}
Lighthill provided the theory to take into account aerodynamic sources. The idea
is that an unsteady flow will produce sound. A simple example is the sound that
is generated when flow passes a string or antenna. The aero-acoustic analogy of
Lighthill is the idea of representing a fluid mechanical process that acts as an
acoustic source by an acoustically equivalent source term.

% He drew an analogy between the non-homogeneous versions of the fluid equations and the classical
% wave equation including the single acoustic source term.

Taking the time derivative of the homogeneous version of the mass conservation
equation and the divergence of the homogeneous version of the momentum
conservation equation, then subtracting one from another, and finally
subtracting from both sides of the equation $c^2 \nabla^2 \rho$, results in the
Lighthill equation
\begin{equation}\label{eq:theory:sound:generation:lighthill}
  \frac{\partial^2 \rho}{\partial t^2} - c^2 \nabla^2 \rho = \nabla \cdot \nabla \cdot \matr{T}
\end{equation}
where $\matr{T}$ is the Lighthill stress tensor
\begin{equation}\label{eq:theory:sound:generation:lighthill-tensor}
  \matr{T} = \rho \vect{u}\vect{u} + \left( p - c^2 \rho \right) \matr{I} - \matr{\tau}
\end{equation}
The right-hand side of the Lighthill equation describes the non-homogeneous
fluid in a finite volume $V$. This area is called the source region or source
field. The left-hand side of the equation describes a homogeneous fluid. This
region is called the sound field and is where the classical wave equation
governs sound propagation.

The Lighthill stress tensor can be considered as a source term of a jet in an
acoustic medium at rest, and consists of three aero-acoustic processes that are
sources of sound \cite{Rienstra2017}. The first are non-linear convective forces
as described by the Reynold stress tensor $\rho \vect{u}\vect{u}$, the second
viscous forces $\tau$ and the third deviations from isentropic behaviour $p - c^2 \rho$.


% A limitation of Lighthill's analogy is that it cannot handle situations where
% feedback occurs because the source field is decoupled from the sound field.


% Lighthill's acoustic analogy has the limitation that the source field is not
% coupled to the acoustic field. That means situations where feedback occurs,
% i.e., the acoustic field modifies the flow field and vice versa, cannot be
% treated.


% The Mach number $M = v / c$ represents the ratio of the flow velocity to the
% local speed of sound.

\subsubsection*{Elementary sources}
Deviations from the wave equation can be considered sources. Sources are
typically classified as a multipole, e.g. a monopole or a dipole. The
aerodynamic sound sources can also be classified as different multipoles, in
case the sources are compact, i.e., the source is small compared to the
wavelength. We will now consider the different aerodynamic sound sources.

The Lighthill equation is an exact reformulation of the fundamental equations.
Equation \eqref{eq:theory:sound:generation:lighthill} is the density-based
variant of the Lighthill equation. Linearisation of the pressure-based variant
results in
\begin{equation}
 \frac{1}{c_0^2}\frac{\partial^2 p'}{\partial t^2} - \frac{\partial^2 p'}{\partial x_i^2} = \frac{\partial}{\partial t} \left( m + \frac{1}{c_0^2}\frac{\partial}{\partial t}\left(p' - c_0^2 \rho' \right) \right) - \frac{\partial f_i}{\partial x_i} + \frac{\partial^2 }{\partial x_i x_j} \left( \rho u_i u_j - \tau_{ij} \right)
\end{equation}
where Cartesian tensor notation is used.
The source term can be split into the following three parts
\begin{align}
  s_1 &= \frac{\partial}{\partial t} \left( m + \frac{1}{c_0^2}\frac{\partial}{\partial t}\left(p' - c_0^2 \rho' \right) \right) \\
  s_2 &= - \frac{\partial f_i}{\partial x_i} \\
  s_3 &= \frac{\partial^2 }{\partial x_i x_j} \left( \rho u_i u_j - \tau_{ij} \right)
\end{align}
which are respectively the monopole, dipole and quadrupole source terms.
% In case a source is compact, i.e., it is small compared to the wavelength, then the source
% distribution can be written as a multipole.
We will now look at these different terms, assuming they are compact non-moving sources.
Considering the Green's function given in the previous section, the general solution is
\begin{equation}
  p'(\vect{x}, t) = \int_V \frac{s(\vect{y}, t-r/c_0)}{4\pi r} \mathrm{d} V
\end{equation}

\subsubsection*{Monopole}
The source mechanisms of the monopole source are mass injection and deviations
from adiabatic behaviour. The sound generated by a compact monopole source due to a mass injection is given by
\begin{equation}
  p'(\vect{x}, t) = \frac{\rho_0 \dot{Q} (t-x/c_0)}{4 \pi x}
\end{equation}
where $\dot{Q}$ is the time-derivative of the total unsteady volume flow
$Q(t) = \int\limits_V q'(\vect{y},t) \mathrm{d}V$.
The sound power is given by
\begin{equation}
  \overline{W_m} = \frac{\rho_0 \overline{\dot{Q}^2}}{4 \pi c_0}
\end{equation}

\subsubsection*{Dipole}
The dipole source term represents an unsteady force
acting on the fluid. An example is a propeller, where the rotating blade forces represent a time-varying force distribution on the fluid.
The sound pressure due to a compact dipole source is given by
\begin{equation}
  p'(\vect{x}, t) = - \frac{\partial}{\partial x_i} \frac{F_i (t-x/c)}{4 \pi x}
\end{equation}
where $F_i(t) = \int\limits_V f_i(\vect{y},t) \mathrm{d}V$ is the total unsteady force acting on the fluid.
The sound power is given by
\begin{equation}
  \overline{W_d} = \frac{\overline{\dot{F}^2}}{12 \pi \rho_0 c_0^3}
\end{equation}

\subsubsection*{Quadrupole}
The quadrupole source term describes noise due to unsteady
momentum transport, $\rho u_i u_j$. The viscous forces $\tau_{ij}$ can be
neglected because they are typically much smaller than the momentum
transportation term and therefore hardly contribute in case of high Reynolds
numbers. The sound pressure due to a quadrupole point source is given by
\begin{equation}
  p'(\vect{x},t) = \frac{\partial^2}{\partial x_i \partial x_j} \left( \frac{Q_{ij} (t-x/c_0)}{4 \pi x}  \right)
\end{equation}
where $Q_{ij}(t) = \int\limits_V T_{ij}(\vect{y},t) \mathrm{d} V$ is the total quadrupole strength and $T_{ij}$ the Lighthill stress tensor.
The sound power is given by
\begin{equation}
  \overline{W_q} = \frac{\epsilon{ij} \overline{\ddot{Q_{ij}}^2} }{\rho_0 c_0^5}
\end{equation}
In this expression $\epsilon_{ij}$ is $1/20\pi$ when $i=j$ and $1/60\pi$ otherwise.
The quadrupole source term is the main noise source of a high-speed jet.

\subsubsection*{Jet noise}
After the Second World War, jet engines began to be used in commercial
airplanes, and their noise emission was a big issue. That eventually led to the
development of Lighthill's theory.
A jet is a stream of fluid that is projected into the surrounding medium
Turbulent mixing of the two fluids causes what is called jet noise. Jet noise is broadband noise, and the
sound power of a jet may scale under certain conditions with Lighthill's $U^8$-law
\begin{equation}
  \overline{W_{j}} \propto \frac{\rho_0 U^8 D^2}{c_0^5}
\end{equation}
where $U$ is the jet speed, and where the characteristic size $D$ would be the diameter of the pipe.

\subsubsection{Source motion}
Source motion affects sound emission and can result in a Doppler shift and
convective amplification. The Doppler shift is a frequency-shift of the signal
and is caused by a time-varying propagation time from source to receiver. The
Doppler shift will be discussed further in \ref{sec:theory:propagation:doppler}.

The source terms that were discussed in this section each represent a multipole,
which has a characteristic directivity. Motion of the source affects sound
emission altering the directivity of the source. This effect is called
convective amplification and is especially noticeable at high source speeds or,
to be more precise, high Mach numbers $M=U/c_0$.

%
%
% % A well known effect related to motion is the Doppler shift, which will be
% % The sound field produced by a moving source is different from that of a fixed source.
% % Motion of a source can result in a Doppler shift as well as strong directivity.
%
% \subsubsection*{Convective wave equation}\todo{keep this for other section?}
% A mean flow has an effect on sound generation and propagation.
% The material derivative $D/Dt$ is defined as
% \begin{equation}
%   \frac{D}{Dt} = \frac{\partial}{\partial t} + \vect{u} \nabla
% \end{equation}
% and is a time derivative of some physical quantity for a portion of a material
% moving with a velocity $\vect{u}$.
%
% Taking the material derivative of the linearised mass conservation equation and
% subtracting the divergence of the linearised momentum conservation equation
% results in the convective wave equation.
% % \begin{equation}
% %   \frac{\partial^2 \rho'}{\partial t^2} + \frac{\partial}{\partial t} \left( 2 \left( \vect{u}_0 \cdot \nabla \right) \rho' \right) + \left( \vect{u}_0 \cdot \nabla \right) \left( \vect{u}_0 \nabla \rho' \right) - \nabla^2 p' = 0
% % \end{equation}
% Assuming an adiabatic process gives
% \begin{equation}
%   \frac{\partial^2 p'}{\partial t^2} + \frac{\partial}{\partial t} \left( 2 \left( \vect{u}_0 \cdot \nabla \right) p' \right) + \left( \vect{u}_0 \cdot \nabla \right)^2 p' - c^2 \nabla^2 p' = 0
% \end{equation}
%
%
% \subsubsection*{Convective amplification}
% Another effect that occurs and is especially relevant for fast-moving sources is
% convective amplification.
%
% \begin{align}
%  p_{m} \propto \left( 1 - M_e \right)^{-2} \\
%  p_{d} \propto \left( 1 - M_e \right)^{-2} \\
%  p_{q} \propto \left( 1 - M_e \right)^{-3} \\
% \end{align}
% Strong directivity can therefore occur for fast-moving sources.
%
%
% \cite{Dowling1976}
%
% % \missingfigure{Convective amplification illustration}

