\section{Generation of sound}

\subsection{Aerodynamic sound sources}\label{sec:theory:sound:aerodynamic}
Classical acoustics provides a method for modeling sound generation due to
vibrating solid boundaries. Aerodynamic sources such as combustion or other
unsteady fluid processes cannot be handled, however. With the development of
jet-powered aircraft a description was needed to model such sources. Lighthill
provided the theory to take into account aerodynamic sources. The idea is that
an unsteady flow will produce sound. A simple example is the sound that is
generated when flow passes a string or antenna.
The aero-acoustic analogy of Lighthill is the idea of representing a fluid
mechanical process that acts as an acoustic source by an acoustically equivalent
source term.

% He drew an analogy between the non-homogeneous versions of the fluid equations and the classical
% wave equation including the single acoustic source term.

Taking the time derivative of the homogeneous version of the mass conservation
equation and the divergence of the homogeneous version of the momentum
conservation equation, then subtracting one from another, and finally
subtracting from both sides of the equation $c^2 \nabla^2 \rho$, results in the
Lighthill equation
\begin{equation}
  \frac{\partial^2 \rho}{\partial t^2} - c^2 \nabla^2 \rho = \nabla \cdot \nabla \cdot \matr{T}
\end{equation}
where $\matr{T}$ is the Lighthill stress tensor
\begin{equation}
  \matr{T} = \rho \vect{u}\vect{u} + \left( p - c^2 \rho \right) \matr{I} - \matr{\tau}
\end{equation}
The right-hand side of the Lighthill equation describes the non-homogeneous
fluid in a finite volume $V$. This area is called the source region or source
field. The left-hand side of the equation is called the sound field, it
describes the homogeneous fluid and this is where the classical wave equation
governs sound propagation.

The Lighthill stress tensor consists of three aero-acoustic processes that are
sources of sound \cite{Rienstra2017}. The first are non-linear convective forces
as described by the Reynold stress tensor $\rho \vect{u}\vect{u}$. The second
viscous forces $\tau$ and the third deviations from a uniform soundspeed
$c_0$ or isentropic behaviour $p' - c_0^2 \rho'$. \todo{requires linearisation}

A limitation of Lighthill's analogy is that it cannot handle situations where
feedback occurs because the source field is decoupled from the sound field.


% Lighthill's acoustic analogy has the limitation that the source field is not
% coupled to the acoustic field. That means situations where feedback occurs,
% i.e., the acoustic field modifies the flow field and vice versa, cannot be
% treated.


% The Mach number $M = v / c$ represents the ratio of the flow velocity to the
% local speed of sound.



\subsection{Elementary sources}

The elementary sources

\missingfigure{Figure of monopole, dipole and quadrupole}

\subsubsection*{Monopole}

Volume source
%
% \begin{equation}
%  \frac{d^2 \rho'}{dt^2}
% \end{equation}

\begin{equation}
 p' \left(\vect{x},t\right) = \frac{\dot{m} \left(\vect{y}, t-r/c\right)}{ 4 \pi r}
\end{equation}
where $\dot{m} = \partial m / \partial t$ is the time derivative of the unsteady mass flow $\dot{m}$.

\subsubsection*{Dipole}

\begin{equation}
 p' \left(\vect{x}, t\right) =
\end{equation}


\subsubsection*{Quadrupole}

Moment

% \subsubsection*{Multipoles and spherical harmonics} % TODO Unless I would write a piece about Ambisonics I have no need for this so could skip it
%
%
% \missingfigure{Multipoles and spherical harmonics}
%
% Ambisonics


\subsection{Moving source}
The sound field produced by a moving source is different from that of a fixed source.

\subsubsection{Convective amplification}
Another effect that occurs and is especially relevant for fast-moving sources is
convective amplification.

\cite{Dowling1976}

\missingfigure{Convective amplification illustration}

