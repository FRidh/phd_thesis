\chapter{Conclusions and future work}\label{chapter:conclusions}

\section{Conclusions}

% TODO: comment Jens, sounds plausible

% Overview paper
Aircraft noise has a negative impact on humans. To further study its impact on
humans a tool was developed to simulate the audible sound field caused by
fly-overs of airplanes. The goal was to develop a tool that can create
auralisations of the current fleet of aircraft.

The auralisation tool consists of an emission synthesiser and a propagation
model based on geometrical acoustics. The emission synthesiser used spectral
modelling synthesis as synthesis strategy and used features extracted from
recordings as input. These features were automatically extracted from a signal
that was obtained after applying an inverse propagation model to a recording.
To verify whether the auralisations sound similar to recordings of the
respective airplane types, a listening test was conducted.

Participants were presented with a paired comparison test. Participants were
able to discriminate between sounds of different aircraft types, in case of both
recordings and auralisations. The similarity ratings between recordings and
auralisations are roughly similar to the similarity ratings between the two
aircraft type. That would imply the auralisations sound like different aircraft
than those that were recorded.

An incorrect reproduction of the blade passing frequency and its harmonics is
the likely cause of the dissimilatity between the auralisations and recordings.
The feature-extraction algorithm was tuned for the relatively small bandwidth of
the Buzz-Saw tones and is therefore underestimating the power and bandwidth of
the blade-passing frequency and harmonics.

Section \ref{sec:introduction:background:auralisation} listed several
auralisation use cases. Whether it is relevant that the auralisations do not
sound exactly similar to the recordings would depend on how the auralisations
would be utilised.

% Turbulence paper
To improve the plausibility of the auralisations
% in situations where the source-receiver distance is typically larger
a method was developed to simulate modulations due to weak atmospheric
turbulence. Spatial and temporal variations in temperature and cause variations
in the soundspeed field. As sound propagates through a turbulent atmosphere, the
variations will cause multiple scattering and result in audible modulations,
known as scintillations. A method was presented for generating sequences of
modulations and applying these to both monochromatic and broadband signals.

Starting with a Helmholtz equation for the sound pressure and a random
soundspeed field, a Rytov approximation to first-order refractive-index
fluctuations was done yielding the sound pressure as a sum of complex phases
with increasing order. The zeroth order represents the unperturbed field and the
first-order represents first-order perturbations of log-amplitude $\chi$ and
phase $S$. Sound that is propagated is modelled as a time-varying channel and
the perturbations are represented by two random sequences, one for the
log-amplitude fluctuations, and another for the phase fluctuations.

The random fluctuations have a Gaussian statistical distribution and a spectrum.
The spectrum is the Fourier transformation of the covariance of the random
temporal fluctuations. To approximate the spectrum of the turbulence a Gaussian
model was used. While other turbulence model are available and could be used,
the Gaussian model was chosen for its simplicity and because its computationally
least demanding. In the Gaussian model the correlation depends on the
correlation length $L$ of the turbulence and the transverse speed $v_{bot}$, and
the variance on distance $d$, correlation length $L$, wavenumber $k$ and
variance of the dynamic refractive-index field. Because of the
frequency-dependency, the fluctuations or modulations were implemented as a
time-variant filter.

Examples were shown where the method is applied to a pure tone and to airplane
auralisations. The log-amplitude fluctuations cause clearly audible modulations,
and the phase fluctuations result in spectral broadening. The spectrograms of
the airplane auralisations show spikes corresponding to amplitude modulations.
They also show an increase in the amount of decorrelation. Furthermore, the
impact of the transverse velocity on the frequency content of the modulations
was demonstrated. While not validated with listening tests, it is the authors
opinion that the method does seem to result in more realistic sounding
auralisations.

\newpage
\section{Future work}
In this section a list of possible future steps is given. The research that was
presented spanned several topics, and therefore the suggested future steps are
quite broad as well. Furthermore, future steps may also depend on the purpose of
e.g. the auralisations as was explained in the introduction.
% The next steps to be taken depend on how auralisations would be used.

\subsubsection*{Investigate human response to aircraft sound}
If one wants to study humans' response to aircraft sound, then the next step is
to test whether auralisations of aircraft are sufficiently similar to the actual
sound of aircraft, with respect to the parameter that is being investigated. For
example, if one is interested in perceived annoyance, then the test should
compare annoyance ratings of the recordings and auralisations. If the conclusion
is that auralisations and recordings give sufficiently similar results, then
that would imply the auralisation method can be used to further study that
aspect of human response within the domain that was considered. If significant
differences would still be found, it would have to be tested what is causing the
differences.

\subsubsection*{Improve estimation and synthesis of blade passing frequency}
From the study that was conducted and presented, it followed that
participants did indeed notice differences between the auralisations and the
recordings. A probable cause of these differences is the estimation of the
power and bandwidth of the blade passing frequency and harmonics. Therefore, a
possible improvement would be to enhance the feature-extraction algorithm to
better estimate these properties. If the amount of fan blades is known, then that
could be used as input to change the behaviour of the algoritm at the blade passing
frequency and its harmonics.

% The most important point to improve is likely the simulation of the
% blade passing frequency and harmonics, and that would require a better
% estimation of their powers and bandwidths. Aside from that, it would seem the
% auralisation tool is capable of delivering plausible auralisations.

\subsubsection*{Develop generic aircraft emission model}
The emission synthesiser uses features determined from one specific event. The
chosen emission synthesis strategy (spectral modeling synthesis with extracted features)
produced auralisations that sound plausible. Therefore, a future step
is to construct a emission model that produces these features as function of
input parameters (e.g. thrust settings).

\subsubsection*{Subjective validation of scintillations model}
The scintillations model is based on an existing statistical model. While the
Gaussian model does not describe the whole turbulence spectrum, it may produce
correct results within a limited domain of the spectrum. In this work the
assumption was made the Gaussian model is sufficient along with other
assumptions (e.g. that the correlation length is much smaller than the Fresnel
zone). A possible future step would be a subjective validation to verify whether
auralisations that use the model sound as realistic as recordings. An open
question is how to conduct such a test, because due the statistical nature of
the turbulence the auralisations and recordings there is a limitation to how
similar they may sound.

\subsubsection*{Different model for describing turbulence spectrum}
As mentioned in the text the basic method for generating scintillations should
also work with other models, like the Von Karman spectrum. The Gaussian model
was chosen for performance reasons, as its simplest, but also because it allowed
further optimisations. Future work could be to develop an optimised algorithm
that utilises other spectra like e.g. the Von Karman spectrum with the use of
expressions given in \cite{Ostashev2015}.

\subsubsection*{Non-isotropic and non-homogeneous atmosphere for the scintillations model}
% \subsubsection*{Height-dependent parameters for the scintillations model}
The current model assumes an isotropic and homogeneous atmosphere. In practice,
the atmosphere is neither and parameters like the correlation length $L$ and the
variance of the refractive-index fluctuations $\langle \mu^2 \rangle$ are
height-dependent \cite{Krasnenko2013}. Ostashev et al. gave formulas for plane
\cite{Ostashev1997b} and spherical \cite{Ostashev1997c} waves propagating
through an anisotropic atmosphere and studies numerically the variances and
correlation functions of the log-amplitude and phase fluctuations
\cite{Ostashev2004}. A future step would be to enhance the scintillations model
to take into account an anisotropic atmosphere.


% TODO check mail Jens with comments about determining parameters

% \subsubsection*{Determine parameters for scintillations model}
% The model that was presented for generating scintillations uses a Gaussian
% correlation function. A Gaussian can be used as an applied filter to get a rough
% approximation to the actual turbulence spectrum within a certain wavenumber
% range. An open question is how to obtain adequate values for the correlation
% length $L$ and the variance of the refractive-index fluctuations $\langle \mu^2
% \rangle$. Values for both can be obtained with a setup as used by Daigle et. al.
% \cite{Daigle1983}, however, with such a setup values can only be obtained at
% relatively low heights.




% \subsection{Develop better indicators for annoyance and sleep disturbance}
