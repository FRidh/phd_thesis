\chapter{Introduction}\label{chapter:introduction}


\section{Background}

\subsection{History of aviation} % TODO Include more names of the events?
On December 17th, 1903, the Wright brothers made the first sustained and
controlled flight of a powered, heavier-than-air, airplane, and in the years
after developed the first practical fixed-wing aircraft, the Flyer III. Their
invention of aircraft controls was a fundamental breakthrough and marked the
beginning of the Pioneer Era of aviation.

The Pioneer Era, lasting until the First World War in August 1914, saw flight
becoming an established technology. Aircraft exhibitions were held,
demonstrations given, and prizes with the intention of encouraging aviation were
offered. Plenty of developments took place in construction, configuration,
controls, propellers and engines. Centres were established for aeronautical
research and flying schools were opened. In 1911 the first aircraft was used for
military purposes by Italy in the Italian-Turkish war and soon after they were
also deployed during the First World War.

In the period between the World Wars airplanes evolved from biplanes made from
mostly wood and fabric to monoplanes made of aluminum. Power of the engines
increased as well. Many aviation firsts occured during this period, like the
first transatlantic flight in 1919, and commercial airlines followed soon on
routes like these.

Development and production continued at an even higher pace during the
the Second World War and saw the development and deployment of jet aircraft
as well as the first turboprop engine that went into mass-production.

After the Second World War commercial aviation grew rapidly, with the first
purpose-built commercial jet airliner scheduled into service in 1952, and the
first sustained and regular service airline operating only 4 years after.
In 1947 a rocket-powered aircraft went through the sound-barrier and
this quickly led to the development of supersonic interceptor aircraft as a
countermeasure against long-range bombers. The development of intercontinental
ballistic missiles and the succesful launch of the Sputnik 1 began the Space
Race increasing again the pace of aeronautical developments.

In 1969 the Apollo 11 landed the first humans on the moon. It was also the year
that both the iconic "Jumbo Jet" Boeing 747 and the Concorde supersonic
passenger airliner had their first flights.
Around this time, commercial airliners started using higher bypass-ratios, resulting in
better fuel economy and less noise. % TODO rewrite part

In the last quarter of the \nth{20} century, the Digital Age, emphasis changed.
Digital computers were used for design and modelling, and digital systems also
started to appear inside the aircraft. Digital fly-by-wire systems improved
manoeuvrability, stability and drag. In-flight management of the flight plan was
being handled by the newly introduced flight management system, reducing the
workload of the crew.

The beginning of the \nth{21} century saw the application of autonomous unmanned
aerial vehicles (UAVs), and the first entirely autonomous flight across the
Atlantic became reality. Furthermore, the solar-powered airplane Solar Impulse 2
completed a circumnavigation of the Earth demonstrating the possibilities of
using renewable energy in an aircraft.

\subsection{Urbanisation, transportation and the impact of aviation}
In 2014 54\% of the world population was living in urban areas. % and this proportion is expected to increase to 66\% by 2050 \cite{UnitedNations2014}.
The world population is growing rapidly and it is expected that the given
percentage increases to 66\% by 2050 \cite{UnitedNations2014}, resulting
in higher urban densities.

People not only live closer to eachother than ever before, but also transport
more than before. Transport allows the spreading of people as well as trade and
is therefore an important aspect for economic growth. People typically commute
to where they work or study, and transport themselves as well for leisure.
Traveling for holidays requires passenger transport, and so does commerce, where
people may need to meet to conduct business. Production and consumption of goods
and products can occur at different locations, thus requiring transport.

While there is a demand for transport, transport also has a negative impact on
our environment. Aviation contributes to climate change and air and noise
pollution. Transport uses most of the world's petroleum creating air pollution
and contributing to global warming and thereby climate change. Despite more
fuel-efficient and less polluting turbofan and turboprop engines, the rapid
growth of air travel contributes to an increase in total pollution attributable
to aviation. In the European Union, greenhouse gas emissions from aviation
increased by 87\% between 1990 and 2006. By 2020, aviation emissions are likely
to more than double from present levels \cite{European2006}.

Furthermore, these environmental issues have the potential to limit the
operation and growth of airports. Indeed, aircraft noise is already a limiting
factor for the capacity of regional and international airports throughout the
world \cite{Zaporozhets2011}.

\subsection{Aircraft noise and human response}
Aircraft noise is noise associated with the operation of airports and in
particular the noise that is caused by aircraft during take-off, flight, or
landing \cite{Zaporozhets2011}. Aircraft noise is a major environmental
constraint for aviation, and is likely to become even more important in the
future considering the densification of cities worldwide.

% how many exposed
In Europe, several millions of people are effected by aircraft noise
\cite{MPDGroupLimited2007}. In for example Switzerland, a country with just over
8 million inhabitants, 3\% (225,000 persons) and 1.3\% (95,000 persons) of the
population are exposed to A-weighted noise levels of above 55 dB by day and above
50 dB by night, respectively \cite{Kirk2009,Schaffer2014}.

% overview of negative effects due to noise
Aircraft noise produced during operations in the vicinity of airports represents
a serious social, ecological, and economic problem. The noise has a negative
impact on people's health, lowers their quality of life, and reduces their
productivity at work \cite{Zaporozhets2011}. Aircraft noise can impact land use
planning \cite{Freestone2010}, cause building restrictions, or result
in additional measures taken like improved insulation of windows. Noise can
affect real estate prices \cite{Theebe2004}, which in turn can cause different
forms of segregation \cite{Bjornskau2005}.

Sound can have a large impact on people's well being. One aspect that has been
extensively investigated is how noise affects noise annoyance. Annoyance is an
unpleasant mental state, and is a term that is used in general for all negative
feelings, such as nuisance, disturbance, unpleasantness and irritation
\cite{Guski1999}. Aircraft noise can have a major impact on noise annoyance,
and there are findings suggesting that people's attitude towards aircraft noise
has changed over the years, rating aircraft noise noiser than before \cite{Babisch2009}.

Another consequence of aircraft noise that is being studied is sleep disturbance
\cite{Michaud2007}. Adequate sleep is essential to one's well being, and
aircraft noise-induced sleep disturbance is therefore often seen as a potential
public health hazard.
How noise affects sleep depends on many factors, the sound pressure level,
duration of the noise, how many sources there are and where the sources are situated,
character of the sound, and thereby also the intermittency of aircraft noise.
Furthermore, there's individual differences such age, sex and noise sensitivity.

\subsection{Aircraft noise mitigation and modelling}
Aircraft noise became a public issue in the 1950s and 1960s. As a consequence
governments enacted legislative controls. Noise regulations typically put a
constraint on the amount of noise that can be produced as measured at certain
locations. Such a constraint can limit the amount of flight operations that are
permitted within a certain time window. As a result, airports may optimise their
flight routes and schedules to reduce noise exposure. Furthermore, aircraft
types may be banned if they're too noisy, providing an incentive for
aircraft manufacturers to develop more quiet aircraft.

% The purpose of aircraft noise mitigation is to reduce noise pollution due to
% aircraft and to the reduce its negative impacts.

The Convention on International Civil Aviation established the International
Civil Aviation Organization (ICAO) \cite{ICAO2017}. This organisation is a
specialised agency of the United Nations and is charged with coordinating and
regulating international air travel. The ICAO adopts standards and guidelines on
matters like navigation, infrastructure and inspection. They also provide
guidelines related to aircraft noise. For example, Doc 9501 describes procedures
for noise emission certification of aircraft \cite{ICAO_9501}.
% Airports may then restrict aircraft based on their emission certification.

% The Effective Perceived Noise Level (EPNL) with unit EPNdB

The European Civil Aviation Conference (ECAC), founded by ICAO and the Council
of Europe, is an intergovernmental organization and is tasked with \say{the
promotion of the continued development of a safe, efficient and sustainable
European air transport system} \cite{ECAC2017}. The ECAC provides Doc 29, which
contains a standardised method for computing noise contours around civil
airports \cite{Doc29_fourth_2016} and is the recommended method for European
Union member states \cite{directive_2002_49_ec}.

As a counter-proposal to a people's initiative to limit the amount of flights
from Z\"{u}rich Airport, the Z\"{u}rich cantonal government proposed to instead limit
the amount of people that were allowed to be highly affected by the aircraft
noise. This counter-proposal, the \say{Zürcher Fluglärm-Index} (ZFI) or Z\"{u}rich
Aircraft Noise Index was accepted \cite{Schaffer2012}. The ZFI is a single
number, representing the amount of persons that are affected by annoyance and/or
sleep disturbance due to aircraft noise related to a single airport.
Exposure-response relations are used for both annoyance during daytime and sleep
disturbance during nighttime.

For the computation of indices like the ZFI it is necessary to obtain
exposure-response relations as well as sound pressure levels that serve as
input. Typically noise levels are predicted on the facades of houses, and a
correction is made to obtain indoor levels. The ZFI considers as input for
determining the amount of sleep disturbed people the A-weighted maximum sound
pressure level $L_{A,max}$ per event and the A-weighted equivalent level
$L_{A,eq}$ over the night period. For annoyance the $L_{A,eq}$ over the daytime
period is used with an additional penalty of 5 dB for the first and last hour.

While exposure-response relations have been determined, the computation of the
ZFI assumes that an event and environment can be entirely described by their
noise levels. However, an assumption or simplication like this disregards the
fact that other aspects than the average or maximum sound pressure level impact
the human response as was mentioned before. Tonal components, for example, are
known to significantly contribute to noise annoyance and therefore improved
metrics may be needed \cite{Sahai2016,Sahai2016b}. Especially with developments
like NASA's X-57 Maxwell Electric Propulsion Airplane
\cite{Moore2012,Beutel2016} which features 14 propellers, each driven by its own
electric engine, and quadcopters \cite{Rizzi2015} that are now commonly used as
drones. Some parameters that affect the sound field and what is perceived by a
listener are the phase relation between propellers, unsteadiness of the sources,
interferences, and fluctuations due to atmospheric turbulence. These parameters
all affect the sound field and what is perceived by a listener. Simple models do
not take into account these parameters, and are therefore insufficient for
determining the impact of the sound on humans.

% \subsection{Aircraft noise prediction models}

% Rizzi2015 - Advances in Distributed Propulsor Acoustic Modeling
% Simple models neglecting unsteadiness of the sources are
% not suitable for use in human annoyance studies that are
% intended to lead to low noise design strategies.


\subsection{Auralisation}\label{sec:introduction:background:auralisation}% of soundscapes}
% \begin{quote}
% Auralisation is the technique of creating audible sound files from numerical data
% \end{quote} \cite{Vorlander2008}.

% Auralisation is a technique to simulate the audible sound of an object or environment.
Auralisation is a method to render audible virtual sound fields \cite{Kleiner1993}.
The method is commonly used in room acoustics to simulate the audible sound inside spaces.
In recent years the method is also used for simulating exterior sound of
cars \cite{Forssen2009,Maillard2012,Pieren2015,Hoffmann2016,Hoffmann2016a},
trains \cite{Pieren2016},
windturbines \cite{Pieren2014,Heutschi2014},
fans \cite{Merino2016} and
aircraft \cite{Arntzen2014a, Rizzi2016a, Rizzi2016}.
The Virtual Acoustic Simulation Technology for Community Noise Technical Working
Group, or in short VASTCON TWG, is a technical working group dedicated to the
auralisation of outdoor sources and environments \cite{Vastcon}.

Sounds are typically created based on some underlying physical model. This
allows the possibility of generating sounds that correspond to specific
situations or conditions and is an important benefit over recordings where it is
not possible to control every parameter. Auralisation is therefore an
interesting tool for studying the impact of a soundscape on humans and the
development of improved descriptors.

The method, essentially a form of virtual reality, can also be used as a
communication tool, for example, when discussing urban development. Noise
contours are typically computed and plotted to show the spatial distribution of
noise. These figures are often hard to interpret for non-specialists, and
they also give no insight in what the noise may sound like.

In 2015 Lelystad airport in The Netherlands was expanding. Residents living
nearby were initially presented with noise contours but struggled with
interpreting them. The Netherlands Aerospace Centre (NLR) was asked to help
residents understand the implications of the airport changes, and so they used
auralisations at townhall meetings \cite{Arntzen2015}.

Generating audible sounds can be done in several ways. Often the entire sound is
synthesised, but this isn't necessarily the case. One could work with existing
sounds and modify these instead. An example would be to take a recording made
outside and in front of a building, and then simulate the sound indoors by applying a
filter to account for the attenuation of walls and windows.
As with noise prediction, emission synthesis and sound propagation are often
separated. This is, not always possible however, e.g. when using a wave-based
method \cite{Hornikx2016,Georgiou2016,Georgiou2016a}.

% Paragraph may still be quite similar to overview paper
Methods that are commonly used for emission synthesis are spectral modelling
synthesis (SMS) and granular synthesis. With granular synthesis small parts of
existing signals are considered, and a new signal is synthesised by combining
these small parts called grains. Grains are often based on measurements, but
that is not necessarily the case. A grain typically corresponds to specific
conditions, for example the speed of the source. Granular
synthesis is a computationally fast method. With spectral modelling synthesis a
signal is created from a superposition of tone and noise components. An
advantage of spectral modelling synthesis is that the synthesis strategy can be
considered separate from the underlying model. Therefore, an emission synthesis
model can be established that relates tonal and spectral components to the
operational state of the source.

% TODO: rewrite from overview paper
% TODO: split partly into aircraft noise modeling
\subsection{Aircraft noise auralisation}
Auralisation is an interesting method and has been used to study future aircraft
types \cite{Rizzi2013,Rizzi2016,Rizzi2016a} and flight procedures
\cite{Sahai2016} through perception-influenced design. Furthermore, the method was used to investigate the perceived
unpleasantness of aircraft flyover noise as a function of certain temporal
parameters \cite{Pate2017}.

Aspects to consider when simulating the sound of aircraft are the different
noise sources on the aircraft, the state of the aircraft (e.g. thrust setting) and thereby the state
of these sources (e.g. frequency engine shaft), as well as the condition of the environment. Main noise
sources on an airplane with turbofan engines are the jet, fan, turbine,
combustor and airframe. The significance of these sources depends on aircraft
type, flight procedure, as well as the position of the source with respect to
the receiver due to directivity of the sources \cite{Bertsch2015}.

% Noise emission prediction models often typically output a spectrum in
% (fractional-)octaves and where no distinction is made between whether the
% emission corresponds to tonal components or noise.

The aircraft emission prediction tools found in the ANOPP-Source Functional
Module of NASA's ANOPP2 \cite{Lopes2016, Tuttle2017} and INSTANT \cite{Sahai2016b}, which is based
on ANOPP, use established models for the noise prediction of the individual
noise sources. The Heidmann model is for example used for fan noise and the
Stone model for jet noise. The Heidmann model in ANOPP models
five sources explicitly, of which three correspond to emission of tones and two
to emission of broadband noise \cite{Arntzen2014a}. The model outputs for each
of these five sources a spectrum in fractional-octaves.
For broadband synthesis in the NASA Auralization Framework (NAF)\cite{Aumann2015}, power of the tonal
components in each band is divided by the amount of tones in that band.
Nowadays the Heidmann model in ANOPP can output the frequencies and amplitudes
of forward and aft radiated fan tones. Only Buzz-Saw noise is still output in
\nicefrac{1}{3}-octaves.

% Both ANOPP2 and INSTANT use spectral modelling
% synthesis to generate an emission signal.

Other models do not describe the contributions from the individual noise
sources or spectral components but merge them together into a single spectrum.
The ECAC Doc29 method \cite{Doc29_fourth_2016} uses the ICAO ANP database and
provides 24 \nicefrac{1}{3}-octave bands, and so does CNOSSOS-EU which has
adopted the \nth{3} edition of Doc29 \cite{Doc29_third_2005}. The Swiss sonAIR
model \cite{Zellmann2016} computes an emission spectrum that is composed of two
source spectra: an engine spectrum and an airframe spectrum. The current Swiss
model, FLULA2, does not consider an emission spectrum but instead uses a database
of immission spectra where propagation effects are already included
\cite{EMPA2010,Schaffer2014}.

Aside from ANOPP2 and INSTANT none of the mentioned models make a distinction
between tonal and broadband noise contributions, thereby making them unfit for the use of
aircraft auralisation which requires explicit knowledge about tonal and noise
contributions.

\subsection{Plausibility of aircraft auralisations and the influence of the atmosphere}
In the last couple of years relatively many papers have been written about
auralisation and aircraft auralisation. Initially basic propagation models were
developed, and then emphasis shifted to the development of emission models. A
common problem with current auralisations is that they can often still sound
artificial because they sound \say{too perfect}.

One cause is that one typically models only the source of interest, neglecting
any background sounds. Aside from impacting how plausible an auralisation
sounds, the lack of such background sounds may also help listeners discriminate
between recordings and auralisations. E.g., in a listening test where recordings
and auralisations of windturbines were compared, cow bells were audible in the
recordings but missing in the auralisations \cite{Pieren2014}.

Another issue is the (un)steadiness of the source and/or
the medium through which the waves propagate. Consider for example a
rotating fan. The fan radiates besides broadband noise also strong tonal
components. Turbulent flow around the fan can cause additional motion of the
fan. This motion will effect the sound that is radiated, causing
modulations of the tonal components. Effects like these are typically not
included in prediction models, and a similar effect was shown to be
important for the assessment of noise annoyance caused by quadcopters
\cite{Rizzi2015}.

In past work it was mentioned that auralisations of (distant) airplanes also
lack a certain randomness or fluctuations in the sound \cite{Arntzen2014a}.
Atmospheric turbulence can cause not only fluctuations of the emitted sound but
can also scatter waves that propagate through the turbulent atmosphere. Temporal
and spatial variations of temperature and wind velocity fields result in
fluctuations of the refractive-index field. Multiple scattering in combination
with these temporal and spatial variations results in log-amplitude and phase
modulations. The log-amplitude fluctuations can often be noticed when listening
to distant aircraft. The phase fluctuations are not perceived by humans
directly, but will result in decorrelation or loss of self-coherence, reducing
the sound pressure level of the signal, and that may be noticeable. Furthermore,
the phase fluctuations can impact the interference between direct and (ground)
reflected sound.

A coherence factor was introduced in earlier work \cite{Shin2006, Arntzen2014b,
Arntzen2014a}. The factor was used to influence the self-coherence of the sound
along a path by introducing phase fluctuations. These fluctuations then resulted
in less pronounced interference dips when multiple propagation paths were
considered. When modelling the different sound sources on a aircraft as separate
sources, their contributions may be Doppler shifted differently, resulting in
audible beating \cite{Rizzi2013}. Decorrelation can reduce the beating.

%
% % TODO: clean up the following
% To enhance future aircraft operations, the Dutch National Aerospace Laboratory (NLR) wants to have a more comprehensive picture on aircraft noise annoyance.
% A cooperation of the NRL with NASA focuses on expanding the annoyance research capabilities \cite{Arntzen2011}. The NLR and NASA use a common setup, the Virtual Community Noise Simulator (VCNS).
% A helmet with visor is used to present a visual simulation and headphones are used to present binaural flyover noise.
%
% At NASA, Rizzi et al. auralised the flyover noise of a hybrid wing body (HWB) aircraft as well as a reference aircraft, similar to a Boeing 777-200ER.
%
% % To predict an aircraft trajectory the flight mechanics are modelled, instead
%
% At RWTH Aachen University a virtual reality room was developed with... \cite{Schroder2010}
%
%
% Figure \ref{fig:background_aircraft_noise_spectrogram_original} shows a spectrogram of a recorded aircraft passage.
%
% \begin{figure}[H]
%         \centering
%         \includegraphics[width=0.9\textwidth]{../figures/spectrogram_of_measurement}
%         \caption{Spectrogram of a measured aircraft passage. Doppler shifted tones can clearly be seen.}
%         \label{fig:background_aircraft_noise_spectrogram_original}
% \end{figure}
%


\subsection{Measurements for sonAIR}\label{sec:introduction:sonair}
Between 2012 and 2016 a new aircraft noise calculation model called sonAIR was
developed at Empa \cite{Zellmann2013,Zellmann2016}. This semi-empirical model is
based on data obtained from flights that occured at Z\"{u}rich airport in 2013 and
2014.

The main dataset consists of sound recordings at various positions nearby the
airport as well as at larger distances. Sound recordings were made at 44.1 kHz with
microphones at a height of 4 meters above ground level and at several locations
simultaneously. The position of the aircraft were recorded in several ways. Two
special cameras were used to determine the aircraft their position when they
were close to the strip. Radar information was available as well. Furthermore,
for a subset of the events additional data was available from the Flight Data
Recorder (FDR). Some examples of data the FDR provides are the trajectory,
engine shaft rotational frequency and configuration of the gear and flaps. All
data was time-synchronised using GPS receivers. Finally, meteorological data was
available.



\newpage
\section{Thesis}


\subsection{Aim}

% \subsection{Aim}

The aim of the thesis is to develop a tool to simulate the audible sound of
airplanes in an urban environment, so that in the future aspects like annoyance
and sleep disturbance due to aircraft noise can be investigated using
auralisations as provided by the tool. The auralisation tool should therefore
provide auralisations that sound sufficiently plausible to investigate these
aspects.

There is a large variety of different aircraft in use nowadays, that could each
be perceived differently. In order to investigate the human response impact of
each of these aircraft, an additional requirement is that the tool should be
able to simulate the audible sound of of the current fleet of airplanes. Instead
of simulating the emission of the aircraft based on (existing) emission models
for the different radiating components, the goal of this work is to investigate
whether plausible sounding auralisations can be made with its emission
properties derived from recordings.

To improve the plausibility of auralisations of aircraft at especially larger
distances, a second goal is to develop a method for incorporating amplitude and
phase fluctuations due to atmospheric turbulence. The method should improve the
plausibility of the auralisations while at the same time have a physical basis
and preferably perform well enough for use in real-time simulators.


%
% The aim of my research is to develop a tool for the auralisation of aircraft noise in
% an urban environment. This tool can then be used to determine better annoyance
% and sleep disturbance indicators and can then be used in conjunction with a noise prediction
% model like sonAIR to give better a better estimate on how many people are affected by aircraft noise and how severe.
%
% The auralisation tool has to support a typical urban situation where shielding
% and reflections may play an important role and should include correct over-head
% directional information. The work is therefore divided in three parts:
%
% \begin{itemize}
%  \item Development of an aircraft source synthesiser
%  \item Development of propagation model
%  \item Installation of reproduction system for auralisation.
%  \end{itemize}

% \section{Thesis structure}
\subsection{Outline}
The thesis is structured as follows.
\newline
\newline
Chapter \ref{chapter:theory} provides an overview of theory required to
understand the concepts that are used throughout the thesis. Discussed are the
basics of sound, signal processing, and aircraft noise emission.
\newline
\newline
Chapter \ref{chapter:tool} describes the auralisation tool that was developed.
The propagation model is explained, as well as an algorithm for extracting
features from recordings that were then used to synthesise aircraft sounds.
\newline
\newline
Chapter \ref{chapter:test} aims to answer the question whether the auralisation
tool provides sufficiently plausible auralisations.
\newline
\newline
Chapter \ref{chapter:turbulence} gives an extensive overview on a novel
algorithm for simulating fluctuations due to atmospheric turbulence and how to
apply this algorithm in auralisations to improve their plausibility.
\newline
\newline
Finally, the work is summarised in Chapter \ref{chapter:conclusions} and future
work is suggested.


\subsection{Scope}
The goal is to develop a tool that can deliver auralisations of airplanes that
sound plausible. Unless mentioned otherwise, the type of aircraft considered are
commercial airliners that have jet engines.

Events that are considered are take-offs relatively close to the airport. In
this case, the airport is Z\"{u}rich Airport as that is where data was gathered.
Take-offs are considered instead of landings because landings can contain
relatively fast-varying frequency components due to thrust corrections made by the pilot.

The auralisation tool considers only source motion and not receiver motion
or a mean motion of the medium. The basic propagation model assumes a homogeneous
and isotropic atmosphere and can therefore not take into account wind or
temperature gradients.

% \begin{itemize}
%
% \end{itemize}


