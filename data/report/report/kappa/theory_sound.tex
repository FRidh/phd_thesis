\section{Sound}

A repetitive variation about a central value of some quantity is called an
oscillation. Oscillations of mechanical nature are vibrations. An
oscillation travelling through a medium and transferring energy is a wave. Sound
is then a mechanical wave travelling through a fluid medium.
In certain fields sound may however refer only to those oscillations that can be perceived by the human brain.
Being a small repetitive perturbation about the barometric
mean pressure of the medium, the fluctuating or dynamic part of the pressure,
denoted sound pressure, is typically many orders smaller than the mean
pressure.

In the \nth{17} century Newton proposed a model for sound waves in elastic media
in his Principia. Already aware that the humidity of the air influences the
speed of sound, Newton assumed an isothermal process for the wave motion and
thereby computed incorrect values for the speed of sound. Laplace gave the
correct derivation of the classical wave equation, describing the wave motion as
a adiabatic process. In the \nth{19} century Kirchoff described the motion of a
rigid body in an ideal fluid and Helmholtz gave a time-independent form of the
wave equation. These were some of the important foundations for the classical
theory of sound.
% A limitation of the developed theory was the lack of sound generation models. In
% the classical theory sound was only generated through a vibrating solid
% boundary. In the 1950s
In this and the following two sections a brief overview is given of sound. Discussed are sound
generation, propagation, and the effect of flow.

% \subsection{Wave equation}\label{sec:theory:sound:wave}
% In his Principia, Newton gave an description of sound and a value of the speed of sound.
The wave equation is a differential equation for describing waves and is used
throughout physics. In the \nth{18} century d'Alembert discovered the
one-dimensional wave equation, and a couple of years later Euler presented the
three-dimensional wave equation. The acoustic wave equation describes the motion
of sound waves and can be derived from the fundamental laws of fluid dynamics \cite{Rienstra2017}.
In this section a brief overview is given of the different wave equations and
expressions that are relevant when considering sound emitted by aircraft.

\subsubsection*{Mass and momentum conservation}
The mass conservation or continuity equation is given by
% Mass conservation equation
\begin{equation}\label{eq:theory:sound:wave:mass}
 \frac{\partial \rho}{\partial t} + \nabla \cdot \left( \rho \vect{u} \right) = m
\end{equation}
with $\rho$ the density of the medium, $t$ the time, $\vect{u}$ the flow velocity
vector, $m$ the mass and $\nabla = \left( \frac{\partial}{\partial
x_1},\frac{\partial}{\partial x_2},\frac{\partial}{\partial x_3} \right)$.
The momentum conservation equation is
% Momentum conservation equation
\begin{equation}\label{eq:theory:sound:wave:momentum}
 \frac{\partial}{\partial t} \rho \vect{u} + \nabla \cdot \left(\matr{P} + \rho \vect{u} \vect{u}  \right) = \vect{f} + m \vect{u}
\end{equation}
where $\vect{u} \vect{u}$ is a dyadic product\footnote{The dyadic product of the
vectors $\vect{a}$ and $\vect{b}$ is the tensor $\vect{a}\vect{b}=a_i b_j$},
$\vect{f}$ the external force density and $\matr{P}$ the fluid stress tensor.
The fluid stress tensor relates the pressure $p$ and the viscous stress tensor
$\matr{\tau}$ by
% Viscous stress tensor
\begin{equation}
  \matr{P} = p \matr{I} - \matr{\tau}
\end{equation}
where $\matr{I}$ is a unit tensor. Viscous stresses are small compared to inertial
forces. Assuming an ideal fluid by ignoring the viscous stresses, and rewriting equation
\eqref{eq:theory:sound:wave:momentum} using equation
\eqref{eq:theory:sound:wave:mass}, we obtain the following form for the momentum conservation equation
\begin{equation}
 \rho \left( \frac{\partial \vect{u}}{\partial t} + \left( \vect{u} \cdot \nabla \right) \vect{u} \right) + \nabla p = \vect{f}
\end{equation}

\subsubsection*{Linearisation}
Sound is a small perturbation of a steady state, and so we can apply
linearisation to obtain a wave equation. Ignoring the source term at the
right-hand side (thus considering the homogeneous solution), the linearised
versions of the mass and momentum equations are given by
% Linearised versions of the homogeneous (= without source terms) mass and momentum equations
\begin{align}
 \frac{\partial \rho'}{\partial t} + \vect{u_0} \cdot \nabla \rho' + \rho_0 \nabla \cdot \vect{u'} = 0  \label{eq:theory:sound:linearisation:mass} \\
% \end{equation}
% \begin{equation}
 \rho_0 \left( \frac{\partial \vect{u'}}{\partial t} + \left(\vect{u_0} \cdot \nabla \right) \vect{u'}\right) + \nabla p' = 0 \label{eq:theory:sound:linearisation:momentum}
\end{align}
with the fluctuating components of the variables denoted with a prime and the
steady components subscripted with a zero.

\subsubsection*{Speed of sound}
Viscosity is neglected and thereby also heat transfer. The fluid is considered
to behave adiabatic and thus the following relation between pressure and density
fluctuations can be used
% Adiabatic relation between pressure and density fluctuations
\begin{equation}
  p' = c^2 \rho'
\end{equation}
where $c$ is the speed of sound
% Speed of sound for adiabatic process
\begin{equation}
  c = \sqrt{ \left( \frac{\partial p}{\partial \rho} \right)_{s} }
\end{equation}
The subscript $s$ indicates an isentropic (constant entropy $s$) or adiabatic process.
In general, the speed of sound is given by $c = \sqrt{\frac{K}{\rho}}$ where $K$
is the bulk modulus of the medium. For ideal gases the bulk modulus is $K=\gamma
p$ where $\gamma=C_p/C_v$ is the ratio of specific heat capacities at constant
pressure $C_p$ and constant volume $C_V$.

\subsubsection*{Classical wave equation}
Taking the time derivative of the linearised mass conservation equation,
subtracting the divergence of the linearised momentum conservation equation
and assuming an adiabatic process results in the classical wave equation
% The convective wave equation considers a flow with a velocity $\vect{u}$. If the velocity is zero, the equation reduces to the classical wave equation
\begin{equation}\label{eq:theory:sound:wave:classic}
 \frac{1}{c^2} \frac{\partial^2 p'}{\partial t^2} - \nabla^2 p' = 0
\end{equation}

\subsubsection*{Harmonic wave}
In acoustics harmonic waves in the following complex form are typically considered
\begin{equation}
  p' (\vect{x}, t) = \hat{p} \left(\vect{x}\right) \exp{\left(j \omega t \right)}
\end{equation}
where $\hat{p} \left(\vect{x}\right)$ is the amplitude of the wave and $\omega=2 \pi f$ the angular frequency.

\subsubsection*{Helmholtz equation}
Inserting (the derivatives of) a harmonic wave in the wave equation results in
the Helmholtz equation
\begin{equation}\label{eq:theory:sound:wave:helmholtz}
 \left( \nabla^2 + k^2 \right) \hat{p} = 0
\end{equation}
where $k=w/c$ is the wavenumber.

\subsubsection*{Plane wave solution}
The solution to the wave equation in one dimension is
\begin{equation}
  p'(x,t) = p'_{+} (t-x/c) + p'_{-} (t+x/c)
\end{equation}
where $p'_{+}$ and $p'_{-}$ are respectively a right and left travelling function.

\subsubsection*{Spherical wave solution}
A solution in three dimensions assuming spherical symmetry is the spherical wave solution
\begin{equation}
  p'(r,t) = \frac{1}{r} p'_{+} (t-r/c) + \frac{1}{r} p'_{-} (t+r/c)
\end{equation}
and looks similar to the plane wave solution. In this expression $r$ is the
distance travelled by the wave. The spherical wave solution represents the sum
of a wave propagating out from the origin and that of a wave propagating towards
the origin. In acoustics only the outgoing wave is typically kept. Contrary to
a plane wave the pressure of a spherical wave decreases with $1/r$.

\subsubsection*{Wave equation with source terms}
The above solutions considered a homogeneous wave equation and therefore do not
take into account any source terms. Linearisation of the mass and momentum equations
would have resulted in a unsteady mass injection $m'$ and unsteady external
force $\vect{f'}$ that correspond to a vibrating solid boundary. The classical wave equation with these terms is
\begin{equation}
   \frac{1}{c^2} \frac{\partial^2 p'}{\partial t^2} - \nabla^2 p' = \frac{\partial m'}{\partial t} - \nabla \cdot \vect{f}'
\end{equation}
In general a source term can be written as a source $s$.

\subsubsection*{Green's function}\label{sec:theory:sound:green}
The simplest possible source is a point source that is represented by a Dirac
delta function $\delta(\vect{x}-\vect{x}_0)$ where $\vect{x}_0$ is the position
of the point source. A Green's function is a solution or impulse response of an
inhomogeneous linear differential equation.
Therefore, the solution of the wave equation with such a point source excitation
is a Green's function. For a harmonic point source the Green's function
$\hat{G}$ should satisfy
\begin{equation}
  \left( \nabla^2 + k^2 \right) \hat{G} = - \delta \left( \vect{x} - \vect{x}_0 \right)
\end{equation}
Note that a Helmholtz equation is used because a harmonic point source is considered.


The solution of this inhomogeneous Helmholtz equation is again an
ingoing and outgoing spherical wave. Because of causality we consider only the
outgoing wave. In free field the Green's function of this outgoing wave is
\begin{equation}\label{eq:theory:sound:green-free-field}
  \hat{G} = \frac{\exp{\left(jkr\right)}}{4 \pi r}
\end{equation}
with $r = \left| \vect{x} - \vect{x}_0 \right|$ the distance from point source to receiver.
% The point source represented by a Dirac delta pulse is a monopole point source of unit amplitude
% and the resulting field is a monopole field.

% Free-field solution
% \begin{equation}
%  G = \frac{1}{4 \pi r} \delta \left( t - \tau - \frac{r}{c} \right)
% \end{equation}


