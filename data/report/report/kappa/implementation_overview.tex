\section{Introduction}

The aim was to develop a tool for the auralisation of airplanes that can cover a
wide range of aircraft that are currently in use. Aircraft are complex noise
sources, with multiple components causing sound emission. The long-term goal is
to develop a generic emission model and the idea is to use measured data
from the sonAIR project for its development. Sound
recordings include propagation effects, and these effects are undesirable when
developing an emission model. This Chapter, which is based on Papers
\ref{paper:overview}, \ref{paper:internoise2016} and \ref{paper:euronoise2015},
describes the tool that was developed.

Removing the propagation effects requires a propagation model, and therefore the
next section discusses first the basic propagation model. The section after
describes the emission synthesis along with a method for determining the
spectral information from airplane recordings. The plausibility of these
auralisations is then compared to that of the recordings in the next chapter by
means of a listening test. Note that parameters that are given in this section
correspond to those that were used for that test.


% As explained in the introduction, the goal is to develop a tool that can
% produce auralisations that are physically as correct as possible, yet at low
% computational cost.
\subsubsection*{Tool overview}
The auralisation tool consists roughly of three parts as shown also in Figure
\ref{fig:implementation:overview}:

\begin{itemize}
\item The first part covers emission synthesis and generates as function of time and
emission angle a signal. Sources are considered to be point sources and the
generated values correspond to their emission at 1 meter from the point source.
Spectral modelling synthesis is chosen as synthesis strategy.

\item The second part describes sound propagation in the far field through a,
for now, isotropic and homogeneous atmosphere. The assumption is made that sound
can be considered a ray.

\item The third part deals with sound immission and more specifically sound immission
encoding. For each source an immission of sound pressure is recorded along with
the angle of incidence as a unit vector. With this data the emission can be
encoded into various formats, for example, as an Ambisonics format using
spherical harmonics, or as monoaural by adding the pressure contributions
ignoring the angles of incidence.
\end{itemize}

\begin{figure}[H]
  \centering
\begin{tikzpicture}[auto, node distance=1cm,>=latex']
\tikzset{
block/.style    = {draw, shape=rectangle, fill=white, minimum height=4em, minimum width=5em, text width=5em, align=center},
}
    % Main nodes
    \node [block]                       (emission)      {Emission};
    \node [block, right=of emission]    (propagation)     {Propagation};
    \node [block, right=of propagation]   (immission)         {Immission};

    % Main edges
    \draw [->]  (emission)      --  (propagation);
    \draw [->]  (propagation)   --  (immission);

\end{tikzpicture}
  \caption{The three separate parts the auralisation tool consists of.}
  \label{fig:implementation:overview}
\end{figure}

% TODO Skip?
The tools mentioned in section \ref{sec:theory:auralisation:software}, as well as other
general-purpose languages, were considered for the implemention.
Eventually the Python general-purpose language was chosen because of its flexibility,
well-tested libraries for computing and signal processing as well as sufficiently good
performance. The auralisation tool was implemented in Python 3.5 \cite{Python}.
Certain computationally intensive routines were implemented in Cython \cite{Behnel2011,Cython}.
Extensive use was made of the Numpy \cite{VanderWalt2011,Numpy}, Scipy \cite{Scipy} and
Pandas \cite{Mckinney2010} libraries. A full implementation of the tool can be
found at \cite{Rietdijk2017d}.
