\section{Overview}

The aim was to develop a tool for the auralisation of airplanes. In this chapter, which is based on Paper \ref{paper:overview}, such a tool is described.
% As explained in the introduction, the goal is to develop a tool that can produce auralisations that are physically as correct as possible, yet at low computational cost.

As with Lighthill's theory we consider a source and sound region. The source is assumed to be composed of one or multiple compact/point sources.
Prominent sources like the engines are indeed small compared to the wavelengths


Sound in the sound region satisfies the homogeneous wave equation.
The
The assumption is made that source and propagation can be 
described independently of each other. 

% \begin{figure}[H]
%         \centering
%         \includegraphics[height=0.4\textheight]{../figures/implementation_overview}
%         \caption{Overview.}
%         \label{fig:implementation_overview}
% \end{figure}

\begin{figure}[H]
  \centering
\begin{tikzpicture}[auto, node distance=1cm,>=latex']
\tikzset{
block/.style    = {draw, shape=rectangle, fill=white, minimum height=4em, minimum width=5em, text width=5em, align=center},
}
    % Main nodes
    \node [block]                       (emission)      {Emission};
    \node [block, right=of emission]    (propagation)     {Propagation};
    \node [block, right=of propagation]   (immission)         {Immission};

    % Main edges
    \draw [->]  (emission)      --  (propagation);
    \draw [->]  (propagation)   --  (immission);

\end{tikzpicture}
  \caption{Overview.}
  \label{fig:implementation:overview}
\end{figure}

The tools mentioned in \ref{sec:theory:auralisation:software} as well as other
general-purpose languages were considered for the implemention.
% TODO further motivation

The auralisation tool was implemented mostly in Python 3.5 \cite{Python} and
Cython \cite{Behnel2011,Cython}. Extensive use was made of the
Numpy\cite{VanderWalt2011,Numpy}, Scipy\cite{Scipy} and
Pandas\cite{Mckinney2010} libraries. A full implementation of the tool can be
found at \cite{Rietdijk2017d}.
