\part{Aircraft noise source synthesiser}

\chapter{Aircraft noise sources and mechanisms}

Noise levels produced by modern aircraft are about 22 dB lower than the first generation jet aircraft. \cite{Zapoozhets2011}
This reduction has been achieved...

\section{Jet noise}

\section{Fan and turbine noise}


\subsection{Buzz-Saw noise}
“Buzz-saw” noise is radiated from a turbofan inlet duct when the fan tip speed 
is supersonic. \cite{McAlpine2007}


\section{Combustion chamber noise}

\section{Airframe noise}

% \section{Propeller and helicopter noise}

\section{Convective 
amplification}\label{sec:source_synthesiser_convective_amplification}


\chapter{Audio synthesis}

\section{Synthesis techniques}
Synthesis techniques have been categorised in different ways. In \cite{JuliusO.Smith} digital synthesis techniques are arranged into four categories:
\begin{itemize}
 \item processed and sampled records
 \item spectral models
 \item physical models
 \item abstract algorithms
\end{itemize}

Misra and Cook produced an overview of synthesis methods for sound designers and 
composers.
According to them the following methods can be used for synthesis of textures 
and soundscapes:
\begin{itemize}
 \item concatenative / granular synthesis
 \item Linear Predictivite Coding (LPC)
 \item stochastic methods
 \item wavelet-based methods
\end{itemize}


\section{Parametric sound representation}
Yercoe et al. present an overview of parametric sound representations \cite{Vercoe1998}.


\section{Synthesising aircraft noise}
The aircraft noise source synthesiser synthesises the sound close to the source.



In earlier work \cite{Arntzen2011} an aircraft trajectory was modelled using 
the flight mechanics instead of a radar track. Based on the flight mechanics 
and several models (Stone, Heidmann, Finkel) to predict spectral components a 
source signal was created.


% One challenging type of noise to include is Buzz-Saw noise. This type of noise 
% is generated by shocks on a fan when the tip is moving supersonic. 
% While work is done at better predicting this type of noise for future turbofans 
% \cite{McAlpine2007} there does not yet exist a model that relates 
% turbofan/aircraft types with conditions to emitted Buzz-Saw noise.

% These models are made for use in noise predictions models, and it may not be so 
% straightforward to use these models for auralisation.
% Therefore \textbf{an overview will be made on how to synthesise an aircraft 
% noise 
% signal}.





\chapter{Data from measurements}
In September 2013 a large measurement campaign was conducted at Empa collecting 
sound recordings of landings as well as take-offs at several 
locations. The measurements resulted in sound recordings, 
positional information of the aircraft, and cockpit data including 
information regarding thrust.

% Along with detailed information regarding flight conditions this 
% opens the 
% possibility to \textbf{develop an empirical model describing Buzz-Saw noise 
as 
% function of aircraft and condition} as well as other types of noise.
% Challenge will be to discriminate between the types of noise as well as 
% atmospheric influences.

\section{Reverse propagation model}
By using a reverse propagation model it is possible to calculate back to the 
source. The resulting signal can be thought of as what one would hear 
when flying along with the aircraft and rotating around the aircraft at one 
meter distance.

An explanation of the reverse propagation model is given 
in \ref{sec:propagation_model_reverse_propagation_model}.


\section{Directivity}\label{sec:source_synthesiser_directivity}
Using the reverse propagation model results in information regarding 
the directivity in, more or less, one plane.
Recordings were made simultaneously at several locations. By combining the 
two the directivity of the source in the lower hemisphere can be 
determined.


% \chapter{Description of the synthesiser}
\chapter{Implementation}

The aircraft noise at the source is considered stationary and is recreated as a 
superposition of signals that are parametrically described, with the parameters 
based on the data obtained from the measurements. A typical aircraft noise 
signal is synthesised using several sine generators and 
bandpass-filtered white noise.

\section{Parameters}




