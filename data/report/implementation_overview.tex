\section{Overview}

In this part the implementation is discussed.

As explained in the introduction, we would like to develop an auralization tool that is physically as correct as possible.

The assumption is made that source and propagation can be 
described independently of each other. 

% \begin{figure}[H]
%         \centering
%         \includegraphics[height=0.4\textheight]{../figures/implementation_overview}
%         \caption{Overview.}
%         \label{fig:implementation_overview}
% \end{figure}

\begin{figure}[H]
  \centering
\begin{tikzpicture}[auto, node distance=1cm,>=latex']
\tikzset{
block/.style    = {draw, shape=rectangle, fill=white, minimum height=4em, minimum width=5em, text width=5em, align=center},
}
    % Main nodes
    \node [block]                       (emission)      {Emission};
    \node [block, right=of emission]    (propagation)     {Propagation};
    \node [block, right=of propagation]   (immission)         {Immission};

    % Main edges
    \draw [->]  (emission)      --  (propagation);
    \draw [->]  (propagation)   --  (immission);

\end{tikzpicture}
  \caption{Overview.}
  \label{fig:implementation:overview}
\end{figure}


The auralisation tool was implemented mostly in Python 3.5 \cite{Python} and
Cython \cite{Behnel2011,Cython}. Extensive use was made of the Numpy\cite{VanderWalt2011,Numpy}, Scipy\cite{Scipy}, Pandas\cite{Mckinney2010},
and Blaze \cite{Blaze} libraries. A full implementation of the tool can be found at
\cite{Rietdijk2017d}
