\section{Auralisation and sound reproduction}

\subsection{Synthesis techniques}
Synthesis techniques have been categorised in different ways. In \cite{JuliusO.Smith} digital synthesis techniques are arranged into four categories:
\begin{itemize}
 \item processed and sampled records
 \item spectral models
 \item physical models
 \item abstract algorithms
\end{itemize}

Misra and Cook produced an overview of synthesis methods for sound designers and 
composers.
According to them the following methods can be used for synthesis of textures 
and soundscapes:
\begin{itemize}
 \item concatenative / granular synthesis
 \item Linear Predictivite Coding (LPC)
 \item stochastic methods
 \item wavelet-based methods
\end{itemize}


\subsection{Parametric sound representation}
Yercoe et al. present an overview of parametric sound representations \cite{Vercoe1998}.

\subsection{Reproduction}


% \subsection{Ambisonics}
%
%
% Increasing the order of ambisonic reproduction allows 3D sound fields to be reconstructed with improved resolution.
%
%
% \subsection{Correct over-head information}
% For the reproduction of the auralisation a system has to be set up. One issue to
% cope with is correct over-head information. There exists a localisation error
% when using an Ambisonics system instead of real sources.
%
% Power et al. conducted subjective localisation tests to determine the spatial accuracy of Ambisonic systems when simulating elevated sources \cite{Power2013}.
% They report the results of subjective localisation tests for virtual sources placed in the vertical plane at various elevations and azimuths, for first, second and third-order ambisonic reproduction over a 16-loudspeaker system.
% An error in elevation angle of 16 to 33 degrees for first and
% second order Ambisonics systems and 10 to 22 degrees for third order
% systems was found \cite{Power2013}.
%
%
% \cite{Power2013}
%
% \subsection{Perception of distance}
%
% \cite{Kearney2012}
