\section{Sound}

\cite{Arntzen2014a, Rienstra2017}

\subsection{Wave equation}\label{sec:theory:sound:wave}

Equation for conservation of mass
\begin{equation}\label{eq:theory:sound:wave:mass}
 \frac{\partial \rho}{\partial t} + \nabla \cdot \rho \vec{u}
\end{equation}
Equation for conservation of momentum
\begin{equation}\label{eq:theory:sound:wave:momentum}
 \frac{\partial}{\partial t} \rho \vec{u} + \nabla \cdot \left( \rho \vec{u} \vec{u}  \right) + \nabla \cdot \bold{P} = \vec{F_n} + m \vec{u}
\end{equation}

% % where $\rho$, $\vec{u}$ and $p$ are the fluid density, velocity and pressure.

$\bold{P} = p \bold{I} - \tau$
viscous tensor $\tau$

Ignoring viscous effects, we can rewrite equation \ref{eq:theory:sound:wave:momentum} using equation \ref{eq:theory:sound:wave:mass} as
\begin{equation}
 \rho \left( \frac{\partial \vec{u}}{\partial t} + \left( \vec{u} \cdot \nabla \right) \vec{u} \right) + \nabla p = \vec{F}
\end{equation}

Linearization.
Mass
\begin{equation}
 \frac{\partial \rho'}{\partial t} + \vec{u_0} \cdot \nabla \rho' + \rho_0 \nabla \cdot \vec{u'} = 0
\end{equation}
Momentum
\begin{equation}
 \rho_0 \left( \frac{\partial \vec{u'}}{\partial t} + \left(\vec{u_0} \cdot \nabla \right) \vec{u'}\right) + \nabla p' = 0
\end{equation}

...
Classical wave equation
\begin{equation}\label{eq:theory:sound:wave:classic}
 \frac{1}{c} \frac{\partial^2 p'}{\partial t^2} - \nabla^2 p' = 0
\end{equation}

...

Helmholtz equation
\begin{equation}\label{eq:theory:sound:wave:helmholtz}
 \nabla^2 \hat{p} + k^2 \hat{p} = 0
\end{equation}
\subsection{Green's function}\label{sec:theory:sound:green}
% Homogeneous
% \begin{equation}
%
% \end{equation}
Non-homogeneous
\begin{equation}
\frac{1}{c^2}\frac{\partial^2 G}{\partial t^2} - \nabla^2 G = \delta \left( \vec{x} - \vec{y} \right) \delta \left(t - \tau \right)
\end{equation}

Free-field solution
\begin{equation}
 G = \frac{1}{4 \pi r} \delta \left( t - \tau - \frac{r}{c} \right)
\end{equation}

with $r = \left| \vec{x} - \vec{y} \right|$


\subsection{Aerodynamic sound sources}\label{sec:theory:sound:aerodynamic}
Lighthill


\subsection{Elementary sources}

The elementary sources 

\missingfigure{Figure of monopole, dipole and quadrupole}

\subsubsection{Monopole}

Volume source
%
% \begin{equation}
%  \frac{d^2 \rho'}{dt^2}
% \end{equation}

\begin{equation}
 p' \left(\vec{x},t\right) = \frac{1}{4\pi} \frac{q \left(\vec{y}, t-r/c\right)}{r}
\end{equation}

\subsubsection{Dipole}

\begin{equation}
 p' \left(\vec{x}, t\right) =
\end{equation}


\subsubsection{Quadrupole}

Moment

\subsubsection{Multipoles and spherical harmonics}\todo{Unless I would write a piece about Ambisonics I have no need for this so could skip it}


\missingfigure{Multipoles and spherical harmonics}

Ambisonics


\subsection{Moving source}

\subsubsection{Convective amplification}

\missingfigure{Convective amplification illustration}

\subsubsection{Doppler shift}

\missingfigure{Doppler shift illustration}



% \chapter{Sound propagation}


\subsection{Atmospheric attenuation}\label{sec:theory_sound_atmospheric_attenuation}
Soundwaves are attenuated due to relaxation processes in the medium.
A model for atmospheric attenuation is given in Part 1 of ISO 9613-1:1993\cite{ISO9613-1}.
The attenuation coefficient $\alpha$, in dB/m, is given by
\begin{align}\label{eq:theory:sound:atmospheric-attenuation}
 \alpha &= 8.686 f^2 \Biggl( \left[ 1.84 \times 10^{-11} \left(\frac{p_r}{p_a}\right)^{-1} \left(\frac{T}{T_0}\right)^{1/2} \right] + \left(\frac{T}{T_0}\right)^{-5/2} \nonumber \\ 
 &\times \Biggl\{ 0.01275 \left[ \exp{\frac{-2239.1}{T}} \right]  \left[f_{r,O} + \frac{f^2}{f_{r,O}} \right]^{-1} \nonumber \\
 &+ 0.1068 \left[ \exp{\frac{-3352.0}{T}} \right] \left[ f_{r,N} + \frac{f^2}{f_{r,N}} \right]^{-1} \Biggr\} \Biggr) 
\end{align}
and is a function of the ambient temperature in kelvin $T$, the reference
temperature $T_0=293.15$ K, the ambient pressure $p_a$ in kilopascal, the
reference pressure $p_r=101.325$ kPa and the relaxation frequencies for oxygen 
$f_{r,O}$ and nitrogen $f_{r,N}$.
The relaxation frequency of oxygen is given by
\begin{equation}
 f_{r,O} = \frac{p_a}{p_r} \left( 24 + 4.04 \cdot 10^4 h \frac{0.02 + h}{0.391 + h}  \right)
\end{equation}
and the relaxation frequency of nitrogen by
\begin{equation}
 f_{r,N} = \frac{p_a}{p_r} \left( \frac{T}{T_0} \right)^{-1/2} \cdot \left( 9 + 280 h \exp{\left\{ -4.170 \left[ \left(\frac{T}{T_0} \right)^{-1/3} -1 \right] \right\} } \right)
\end{equation}
Both depend on the molar concentration of water vapour $h$, given by
\begin{equation}
 h = h_r  \frac{p_{sat}}{p_a}
\end{equation}
The molar concentration of water vapour is a function of the saturation pressure
\begin{equation}
 p_{sat} = 10^C \cdot p_r
\end{equation}
where 
\begin{equation}
 C = -6.8346 \cdot \left( \frac{T_{01}}{T} \right)^{1.261}  + 4.6151
\end{equation}
In this equation $T_{01}$ is the triple-point isotherm temperature of 273.16 K.

The standard furthermore mentions the following expression for the speed of sound
\begin{equation}
c = 343.2 \left( \frac{T}{T_0} \right)
\end{equation}

\begin{figure}[H]
        \centering
        \includegraphics[]{../figures/generated/sound/attenuation}
        \caption{Atmospheric attenuation as function of frequency for a standard atmosphere.}
        \label{fig:theory:sound:attenuation}
\end{figure}

\subsection{Descriptors}


\newpage
\subsection{Reflections}

\subsubsection{Impedance}
Several models exist for the prediction of the impedance of a surface.
Attenborough made a comparison of impedance models\cite{Attenborough2011a}.


One such models is the empirical 1-parameter model by Delaney and Bazley which
only depends on the flow resistivity $\sigma$ aside from the frequency $f$.
According to this model, the normalised impedance $Z$ is given by
\begin{equation}\label{eq:theory:sound:impedance:db}
 Z = 1 + 9.08 \left( \frac{1000f}{\sigma}\right)^{-0.75} - 11.9 j \left( \frac{1000f}{\sigma}\right)^{-0.73}
\end{equation}

Another example of a model is the 2-parameter model by Attenborough. In this model, the impedance is given by
\begin{equation}\label{eq:theory:sound:impedance:att}
 Z = \frac{\left( 1-j\right) \sqrt{\sigma/f}}{\sqrt{\pi \gamma_0 \rho_0}} - \frac{jc\alpha}{8 \pi \gamma_0 f}
\end{equation}
and depends on the speed of sound in air $c_0$, the density of air $\rho_0$, $\alpha$, $\gamma_0$ and again the flow resistivity $\sigma$ and frequency $f$.



\subsubsection{Reflection coefficient}

The plane wave reflection coefficient is given by
\begin{equation}\label{eq:theory:sound:reflection:plane}
 R = \frac{Z\cos{\theta}-1}{Z\cos{\theta}+1}
\end{equation}

The spherical reflection coefficient is given by
\begin{equation}\label{eq:theory:sound:reflection:spherical}
 Q = R \left(1 - R \right) F
\end{equation}
with 
\begin{equation}
 F = 1 - j \sqrt{ \pi} w e^{-w^2} \mathrm{erfc} \left( j w \right) 
\end{equation}
and
\begin{equation}
 w = \sqrt{-j k r  \left( 1 + \frac{1}{Z} \cos{\theta} - \sqrt{1 - \left( \frac{1}{Z} \right)^2} \sin{\theta} \right) }
\end{equation}




\missingfigure{SPL relative to free field comparison}



\subsubsection{Image source method}

\missingfigure{Illustration of ISM}


\subsubsection{Ground effect}

A prominent reflection is the reflection with the ground. When the 
reflecting surface is sufficiently hard the Lloyd's 
mirror effect can be heard.
The spectogram in figure \ref{fig:propagation_reflections_mirror_effect} shows the Lloyd's mirror effect as well.

% \begin{figure}[H]
%         \centering
%         \includegraphics[width=0.9\textwidth]{../figures/ipynb/theory_reflections_mirror_effect/figure1}
%         \caption{Spectrogram of a white noise source flying at a height of 100 meters over an acoustically hard surface. The spectrogram clearly shows the Lloyd's mirorr effect.}
%         \label{fig:theory_reflections_mirror_effect}
% \end{figure}




