\section{Sound}

A repetitive variation about a central value of some quantity is called an
oscillation. Oscillations of mechanical nature are vibrations. An
oscillation travelling through a medium and transferring energy is a wave. Sound
is then a mechanical wave travelling through a fluid medium. Furthermore, only
those oscillations that can be perceived by the human brain are typically
considered sound.
Being a small repetitive perturbation about the barometric
mean pressure of the medium, the fluctuating or dynamic part of the pressure,
denoted sound pressure, is typically many orders smaller than the mean
pressure.

In the \nth{17} century Newton proposed a model for sound waves in elastic media
in his Principia. Already aware that the humidity of the air influences the
speed of sound, Newton assumed an isothermal process for the wave motion and
thereby computed incorrect values for the speed of sound. Laplace gave the
correct derivation of the classical wave equation, describing the wave motion as
a adiabatic process. In the \nth{19} century Kirchoff described the motion of a
rigid body in an ideal fluid and Helmholtz gave a time-independent form of the
wave equation. These were some of the important foundations for the classical
theory of sound.
% A limitation of the developed theory was the lack of sound generation models. In
% the classical theory sound was only generated through a vibrating solid
% boundary. In the 1950s
In this section a brief overview is given of sound.



\subsection{Wave equation}\label{sec:theory:sound:wave}
% In his Principia, Newton gave an description of sound and a value of the speed of sound.
The wave equation is a differential equation for describing waves and is used
throughout physics. In the \nth{18} century d'Alembert discovered the
one-dimensional wave equation, and a couple of years later Euler presented the
three-dimensional wave equation. The acoustic wave equation describes the motion
of sound waves and can be derived from the fundamental laws of fluid dynamics \cite{Arntzen2014a, Rienstra2017}.

\subsubsection*{Mass and momentum conservation}
The mass conservation or continuity equation is given by
% Mass conservation equation
\begin{equation}\label{eq:theory:sound:wave:mass}
 \frac{\partial \rho}{\partial t} + \nabla \cdot \rho \vect{u} = m
\end{equation}
with $\rho$ the density of the medium, $t$ the time, $\vect{u}$ the flow velocity
vector, $m$ the mass and $\nabla = \left( \frac{\partial}{\partial
x_1},\frac{\partial}{\partial x_2},\frac{\partial}{\partial x_3} \right)$.
The momentum conservation equation is
% Momentum conservation equation
\begin{equation}\label{eq:theory:sound:wave:momentum}
 \frac{\partial}{\partial t} \rho \vect{u} + \nabla \cdot \left(\matr{P} + \rho \vect{u} \vect{u}  \right) = \vect{f} + m \vect{u}
\end{equation}
where $\vect{u} \vect{u}$ is a dyadic product, $\vect{f}$ the external force
density and $\matr{P}$ the fluid stress tensor. The fluid stress tensor relates the pressure $p$ and the viscous stress tensor $\matr{\tau}$by
% Viscous stress tensor
\begin{equation}
  \matr{P} = p \matr{I} - \matr{\tau}
\end{equation}
where $\matr{I}$ is a unit tensor. Viscous stresses are small compared to inertial
forces. Ignoring them, and rewriting equation
\ref{eq:theory:sound:wave:momentum} using equation
\ref{eq:theory:sound:wave:mass}, we obtain
\begin{equation}
 \rho \left( \frac{\partial \vect{u}}{\partial t} + \left( \vect{u} \cdot \nabla \right) \vect{u} \right) + \nabla p = \vect{f}
\end{equation}
\subsubsection*{Linearisation}
Sound is a small perturbation of a steady state, and so we can apply linearisation to obtain a wave equation.
Ignoring the source term, the linearised versions of the mass and momentum equations are given by
% Linearised versions of the homogeneous (= without source terms) mass and momentum equations
\begin{align}
 \frac{\partial \rho'}{\partial t} + \vect{u_0} \cdot \nabla \rho' + \rho_0 \nabla \cdot \vect{u'} = 0 \\
% \end{equation}
% \begin{equation}
 \rho_0 \left( \frac{\partial \vect{u'}}{\partial t} + \left(\vect{u_0} \cdot \nabla \right) \vect{u'}\right) + \nabla p' = 0
\end{align}
with the fluctuating components of the variables denoted with a prime. Viscosity
is neglected and thereby also heat transfer. The fluid is considered to behave
adiabatic and therefore the following relation between pressure and density
fluctuations can be used
\begin{equation}
  p' = c^2 \rho'
\end{equation}
where $c$ is the speed of sound for an adiabatic process, i.e., a process at constant entropy $s$
\begin{equation}
  c = \sqrt{ \left( \frac{\partial p}{\partial \rho} \right)_{s} }
\end{equation}
In general, the speed of sound is given by $c = \sqrt{\frac{K}{\rho}}$ where $K$
is the bulk modulus of the medium. For ideal gases the bulk modulus is $K=\gamma
p$ where $\gamma=C_p/C_v$ is the ratio of specific heat capacity at constant
pressure $C_p$ and volume $C_V$.

\todo{combine into wave equation}
...
Classical wave equation
\begin{equation}\label{eq:theory:sound:wave:classic}
 \frac{1}{c} \frac{\partial^2 p'}{\partial t^2} - \nabla^2 p' = 0
\end{equation}
...


\subsubsection*{Helmholtz equation}

\begin{equation}
  p' (t, \vect{x}) = \hat{p} \left(\vect{x}\right) \exp{j \omega t}
\end{equation}


\begin{equation}\label{eq:theory:sound:wave:helmholtz}
 \nabla^2 \hat{p} + k^2 \hat{p} = 0
\end{equation}


\subsection{Convective wave equation}
A mean flow has an effect on sound generation and propagation.
The material derivative $D/Dt$ is defined as
\begin{equation}
  \frac{D}{Dt} = \frac{\partial}{\partial t} + \vect{u} \nabla
\end{equation}
and is a time derivative of some physical quantity for a portion of a material moving with a velocity $\vect{u}$.





\subsubsection*{Green's function}\label{sec:theory:sound:green}
% Homogeneous
% \begin{equation}
%
% \end{equation}
Non-homogeneous
\begin{equation}
\frac{1}{c^2}\frac{\partial^2 G}{\partial t^2} - \nabla^2 G = \delta \left( \vect{x} - \vect{y} \right) \delta \left(t - \tau \right)
\end{equation}

Free-field solution
\begin{equation}
 G = \frac{1}{4 \pi r} \delta \left( t - \tau - \frac{r}{c} \right)
\end{equation}

with $r = \left| \vect{x} - \vect{y} \right|$


\subsection{Aerodynamic sound sources}\label{sec:theory:sound:aerodynamic}
The classic wave equation cannot describe aerodynamic sources. Lighthill
provided the theory to take into account aerodynamic sources. The aero-acoustic
analogy of Lighthill is the idea of representing a fluid mechanical process that
acts as an acoustic source by an acoustically equivalent source term.

He drew an analogy
between the non-homogeneous versions of the fluid equations and the classical
wave equation including the single acoustic source term.

Taking the time derivative of the homogeneous version of the mass conservation
equation (eq. \ref{eq:theory:sound:wave:mass}) and the divergence of the
homogeneous version of the momentum conservation equation (eq.
\ref{eq:theory:sound:wave:momentum}), then subtracting one from another and
finally subtracting from both sides $c^2 \nabla^2 \rho$ results in the Lighthill
equation
\begin{equation}
  \frac{\partial^2 \rho}{\partial t^2} - c^2 \nabla^2 \rho = \nabla^2 \matr{T}
\end{equation}
where $\matr{T}$ is the Lighthill stress tensor
\begin{equation}
  \matr{T} = \left( \left( p - c^2 \rho) \matr{I} - \matr{\tau}  \right)
\end{equation}

Lighthill's acoustic analogy has the limitation that the source field is not
coupled to the acoustic field. That means situations where feedback occurs,
i.e., the acoustic field modifies the flow field and vice versa, cannot be
treated.

\subsection{Elementary sources}

The elementary sources 

\missingfigure{Figure of monopole, dipole and quadrupole}

\subsubsection*{Monopole}

Volume source
%
% \begin{equation}
%  \frac{d^2 \rho'}{dt^2}
% \end{equation}

\begin{equation}
 p' \left(\vect{x},t\right) = \frac{1}{4\pi} \frac{q \left(\vect{y}, t-r/c\right)}{r}
\end{equation}

\subsubsection*{Dipole}

\begin{equation}
 p' \left(\vect{x}, t\right) =
\end{equation}


\subsubsection*{Quadrupole}

Moment

\subsubsection*{Multipoles and spherical harmonics}\todo{Unless I would write a piece about Ambisonics I have no need for this so could skip it}


\missingfigure{Multipoles and spherical harmonics}

Ambisonics


\subsection{Moving source}

\subsubsection{Convective amplification}

\missingfigure{Convective amplification illustration}

\subsubsection{Doppler shift}

\missingfigure{Doppler shift illustration}



% \chapter{Sound propagation}


\subsection{Atmospheric attenuation}\label{sec:theory_sound_atmospheric_attenuation}
Soundwaves are attenuated due to relaxation processes in the medium.
A model for atmospheric attenuation is given in Part 1 of ISO 9613-1:1993\cite{ISO9613-1}.
The attenuation coefficient $\alpha$, in dB/m, is given by
\begin{align}\label{eq:theory:sound:atmospheric-attenuation}
 \alpha &= 8.686 f^2 \Biggl( \left[ 1.84 \times 10^{-11} \left(\frac{p_r}{p_a}\right)^{-1} \left(\frac{T}{T_0}\right)^{1/2} \right] + \left(\frac{T}{T_0}\right)^{-5/2} \nonumber \\ 
 &\times \Biggl\{ 0.01275 \left[ \exp{\frac{-2239.1}{T}} \right]  \left[f_{r,O} + \frac{f^2}{f_{r,O}} \right]^{-1} \nonumber \\
 &+ 0.1068 \left[ \exp{\frac{-3352.0}{T}} \right] \left[ f_{r,N} + \frac{f^2}{f_{r,N}} \right]^{-1} \Biggr\} \Biggr) 
\end{align}
and is a function of the ambient temperature in kelvin $T$, the reference
temperature $T_0=293.15$ K, the ambient pressure $p_a$ in kilopascal, the
reference pressure $p_r=101.325$ kPa and the relaxation frequencies for oxygen 
$f_{r,O}$ and nitrogen $f_{r,N}$.
The relaxation frequency of oxygen is given by
\begin{equation}
 f_{r,O} = \frac{p_a}{p_r} \left( 24 + 4.04 \cdot 10^4 h \frac{0.02 + h}{0.391 + h}  \right)
\end{equation}
and the relaxation frequency of nitrogen by
\begin{equation}
 f_{r,N} = \frac{p_a}{p_r} \left( \frac{T}{T_0} \right)^{-1/2} \cdot \left( 9 + 280 h \exp{\left\{ -4.170 \left[ \left(\frac{T}{T_0} \right)^{-1/3} -1 \right] \right\} } \right)
\end{equation}
Both depend on the molar concentration of water vapour $h$, given by
\begin{equation}
 h = h_r  \frac{p_{sat}}{p_a}
\end{equation}
The molar concentration of water vapour is a function of the saturation pressure
\begin{equation}
 p_{sat} = 10^C \cdot p_r
\end{equation}
where 
\begin{equation}
 C = -6.8346 \cdot \left( \frac{T_{01}}{T} \right)^{1.261}  + 4.6151
\end{equation}
In this equation $T_{01}$ is the triple-point isotherm temperature of 273.16 K.

The standard furthermore mentions the following expression for the speed of sound
\begin{equation}
c = 343.2 \left( \frac{T}{T_0} \right)
\end{equation}

\begin{figure}[H]
        \centering
        \includegraphics[]{../figures/generated/sound/attenuation}
        \caption{Atmospheric attenuation as function of frequency for a standard atmosphere.}
        \label{fig:theory:sound:attenuation}
\end{figure}

\subsection{Descriptors}


\newpage
\subsection{Reflections}

\subsubsection{Impedance}
Several models exist for the prediction of the impedance of a surface.
Attenborough made a comparison of impedance models\cite{Attenborough2011a}.


One such models is the empirical 1-parameter model by Delaney and Bazley which
only depends on the flow resistivity $\sigma$ aside from the frequency $f$.
According to this model, the normalised impedance $Z$ is given by
\begin{equation}\label{eq:theory:sound:impedance:db}
 Z = 1 + 9.08 \left( \frac{1000f}{\sigma}\right)^{-0.75} - 11.9 j \left( \frac{1000f}{\sigma}\right)^{-0.73}
\end{equation}

Another example of a model is the 2-parameter model by Attenborough. In this model, the impedance is given by
\begin{equation}\label{eq:theory:sound:impedance:att}
 Z = \frac{\left( 1-j\right) \sqrt{\sigma/f}}{\sqrt{\pi \gamma_0 \rho_0}} - \frac{jc\alpha}{8 \pi \gamma_0 f}
\end{equation}
and depends on the speed of sound in air $c_0$, the density of air $\rho_0$, $\alpha$, $\gamma_0$ and again the flow resistivity $\sigma$ and frequency $f$.

%
% Table \ref{tab:theory:sound:impedance:flow-resistivity} lists co
%
%
% \begin{tabular}{l c}
% Team              & P & W & D & L & F  & A & Pts \\
% \hline
% Manchester United & 6 & 4 & 0 & 2 & 10 & 5 & 12  \\
% Celtic            & 6 & 3 & 0 & 3 &  8 & 9 &  9  \\
% Benfica           & 6 & 2 & 1 & 3 &  7 & 8 &  7  \\
% FC Copenhagen     & 6 & 2 & 1 & 3 &  5 & 8 &  7  \\
% \end{tabular}



\subsubsection{Reflection coefficient}

The plane wave reflection coefficient is given by
\begin{equation}\label{eq:theory:sound:reflection:plane}
 R = \frac{Z\cos{\theta}-1}{Z\cos{\theta}+1}
\end{equation}

The spherical reflection coefficient is given by
\begin{equation}\label{eq:theory:sound:reflection:spherical}
 Q = R \left(1 - R \right) F
\end{equation}
with 
\begin{equation}
 F = 1 - j \sqrt{ \pi} w e^{-w^2} \mathrm{erfc} \left( j w \right) 
\end{equation}
and
\begin{equation}
 w = \sqrt{-j k r  \left( 1 + \frac{1}{Z} \cos{\theta} - \sqrt{1 - \left( \frac{1}{Z} \right)^2} \sin{\theta} \right) }
\end{equation}




% \missingfigure{SPL relative to free field comparison}



\subsubsection{Image source method}
\todo{while implemented I don't consider multiple reflections}

\missingfigure{Illustration of ISM}


\subsubsection{Ground effect}

A prominent reflection is the reflection with the ground. When the 
reflecting surface is sufficiently hard the Lloyd's 
mirror effect can be heard.
The spectogram in figure \ref{fig:propagation_reflections_mirror_effect} shows the Lloyd's mirror effect as well.

% \begin{figure}[H]
%         \centering
%         \includegraphics[width=0.9\textwidth]{../figures/ipynb/theory_reflections_mirror_effect/figure1}
%         \caption{Spectrogram of a white noise source flying at a height of 100 meters over an acoustically hard surface. The spectrogram clearly shows the Lloyd's mirorr effect.}
%         \label{fig:theory_reflections_mirror_effect}
% \end{figure}




