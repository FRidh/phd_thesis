\chapter{Conclusions and future work}\label{chapter:conclusions}


\subsubsection{Auralization paper}\todo{rewrite and merge}
In order to investigate the impact of aircraft sound on humans a tool was
developed to auralise aircraft sound. The goal was to develop a tool that can
create auralisations that sound similar to the target aircraft and is capable of
creating auralisations of the current fleet of aircraft.

A propagation model based on geometrical acoustics was developed and the
emission synthesis was based on features extracted from recordings with the use
of an inverse propagation model and a feature extraction algorithm. A listening
test was conducted to determine whether the chosen features are sufficient for
creating auralisations that sound similar to the actual aircraft.

From the listening test it follows that listeners can discriminate between
sounds of different aircraft types, in case of both recordings and
auralisations. The similarity ratings between recordings and auralisations are
similar to the similarity ratings between the two aircraft type, implying the
auralisations sound like different aircraft than those that were recorded.
The dissimilarity between the auralisations and recordings is believed to be
caused by the incorrect reproduction of the blade passing frequency and its
harmonics caused by an underestimation of the power and bandwidth during the
feature-extraction.

Section \ref{sec:introduction:background:auralisation} listed several
auralisation use cases. Whether it is relevant that the auralisations do not
sound exactly similar to the recordings would depend on how the auralisations
would be utilised.

\subsubsection{Turbulence paper}\todo{rewrite and merge}
To improve the plausibility of the auralisations
% in situations where the source-receiver distance is typically larger
a method was developed to simulate for auralisations modulations
due to atmospheric turbulence. Fluctuations in the refractive-index field due to
variations in temperature and wind affects sound propagation and causes audible
modulations. A method was presented for generating sequences of modulations and
applying these to monochromatic as well as broadband signals.

A Rytov approximation to first-order refractive-index flucutuations results in a complex
phase which we can write as a log-amplitude $\chi$ and phase $S$ fluctuation.
The propagating sound is modelled as a time-varying channel where we consider
two sequences, one for the log-amplitude fluctuations, and another for the phase
fluctuations.

The fluctuations are frequency-dependent and therefore a filter was designed to take that into account.
% A Gaussian applied filter was used to model the atmospheric turbulence because of its simplicity and because it is computationally least demanding.
A Gaussian turbulence spectrum was considered, but the general method can be
used with other turbulence spectra as well. The Von Karman spectrum describes
real turbulence spectra typically better than the Gaussian spectrum, however,
the Von Karman spectrum is computationally much more demanding.
% The Gaussian model also allows implementing the phase fluctuation with a Variable Delay Line (VDL).

Examples are shown where the method is applied to a tone and to an aircraft
auralization. The aircraft auralization spectrogram shows several spikes
corresponding to amplitude modulations as well as an increase in the amount of
decorrelation. Furthermore the transverse velocity dependence on the frequency
content of the modulations is demonstrated.

According to the authors the method results in more realistic sounding
auralizations, but this has not been validated yet with listening tests. The
implementation of the model that was used in this paper to generate the figures
can be found at \cite{Rietdijk2016}.

\section{Future work}
Future steps to be taken depends strongly on how auralisations would
be used.
% The next steps to be taken depend on how auralisations would be used.

\subsubsection*{Investigate human response to aircraft sound}
If one wants to study humans' response to aircraft sound, then the next step is
to test whether auralisations of aircraft are sufficiently similar to the actual
sound of aircraft, with respect to the parameter that is being investigated. If
the conclusion is that auralisations and recordings give sufficiently similar
results, then that would imply the auralisation method can be used to further
study that aspect of human response within the domain that was considered. If
significant differences would still be found, it would have to be tested what is
causing the differences.

\subsubsection*{Improve estimation and synthesis of blade passing frequency}
From the study that was conducted and presented in this work, it followed that
participants did indeed notice differences between the auralisations and the
recordings. These differences are believed to be caused by the estimation of the
power and bandwidth of the blade passing frequency and harmonics. Therefore, a
future would be to enhance the feature-extraction algorithm to better estimate
these properties.

% The most important point to improve is likely the simulation of the
% blade passing frequency and harmonics, and that would require a better
% estimation of their powers and bandwidths. Aside from that, it would seem the
% auralisation tool is capable of delivering plausible auralisations.

\subsubsection*{Develop generic aircraft emission model}
Because the auralisations appear to sound plausible it is concluded that the strategy\todo{what strategy? recording, backprop, ...}
utilised works and therefore


% \subsection{Develop better indicators for annoyance and sleep disturbance}
